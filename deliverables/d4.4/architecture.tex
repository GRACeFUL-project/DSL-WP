The software stack of the GRACeFUL project consists of
%
a visual editor frontend,
%
a network layer,
%
a GCM component library,
%
a DSL called GRACe,
%
a middleware called haskelzinc, and
%
a choice of external constraint solver.
%
In this section we briefly describe the different layers to explain
the context.

\subsection*{Visual Editor}

The top layer of the software stack is the visual editor.
%
It provides a graphical user interface where the user can build GCMs
as a graphical map from available components.
%
The visual editor is implemented in the untyped functional language
JavaScript, using the Data Driven Documents library (D3.js).
%

The code for the visual editor can be found in the 
\href{https://github.com/GRACeFUL-project}{GRACeFUL-project} GitHub
repository
\href{https://github.com/GRACeFUL-project/GRACeFULEditor}{GRACeFULEditor}.

\subsection*{Communication with visual editor}

Communication between the visual editor and the GRACe layer takes
place through a RESTful Web service written in Haskell.
%
JSON objects are sent between the two layers via requests to this
service and handled on both ends.

The GRACe web service is available as the \texttt{RestAPI} executable in the
\href{https://github.com/GRACeFUL-project/GRACe}{GRACe} repository.

\subsection*{GCM component libraries}

The visual interface allows the user to access a chosen library of GCM
components.
%
These components are written in GRACe, and each component has a
corresponding JSON interface which is sent to the visual editor when
the user requests the library in question.

\subsection*{GRACe}

GRACe is a domain specific language embedded in Haskell.
%
It is used to express GRACeFUL concept maps (GCMs) and GCM library
components.
%
GRACe programs representing GCMs are compiled to haskelzinc constraint
programs, and the resulting solutions are passed back to GRACe.

\subsection*{Haskelzinc}

Haskelzinc is a Haskell interface to the MiniZinc constraint
programming language.
%
It provides
\begin{itemize}
\item a Haskell abstract syntax tree for the MiniZinc language, with
  which one can represent MiniZinc models in Haskell
\item a human-friendly DSL for building MiniZinc model representations
\item a pretty printer to print the representation of a MiniZinc model
  in MiniZinc
\item a parser that returns a representation of the solutions obtained
  by running the MiniZinc model
\item a set of functions useful for building a custom FlatZinc
  solutions parser.
  %
  An additional module gives the possibility to directly obtain the
  solutions of a MiniZinc finite domain model.
  %**TODO explain what that means
%   Option for interactive interface is provided, as well as choice
%   between two solvers: the G12/FD built-in solver of FlatZinc and
%   choco3.
\end{itemize}

The latest version of haskelzinc (currently 0.3) is available from
\url{https://hackage.haskell.org/package/haskelzinc}.

\subsection*{MiniZinc}

MiniZinc is a constraint programming language in which constraint satisfaction
and optimization problems can be modeled independently of constraint solvers.

MiniZinc models are compiled into FlatZinc, a low-level solver input language.
There are a variety of solvers that can be used to solve problems stated in
FlatZinc.

Information on how to install and use MiniZinc is available at
\url{www.minizinc.org}.
