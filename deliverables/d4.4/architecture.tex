The software stack of the GRACeFUL project consists of
%
a visual editor frontend,
%
a network layer,
%
a GCM component library,
%
a DSL called GRACe,
%
a middleware called haskelzinc, and
%
a choice of external constraint solver.
%
In this section we briefly describe the different layers and how testing and
verification is performed on each layer.

\subsubsection*{Visual Editor}

The top layer of the software stack is the visual editor.
%
It provides a graphical user interface where the user can build GCMs
as a graphical map from available components.
%
The visual editor is implemented in the untyped functional language
JavaScript, using the Data Driven Documents library (D3.js).
%

The code for the visual editor can be found in the
\href{https://github.com/GRACeFUL-project}{GRACeFUL-project} GitHub
repository
\href{https://github.com/GRACeFUL-project/GRACeFULEditor}{GRACeFULEditor}.

\todo{Is there anything to say about testing here? Maybe refer to user testing
  related deliverable/stuff}

The visual editor is described in deliverabe D3.3 including how it can
be
accessed\footnote{\url{http://vocol.iais.fraunhofer.de/graceful-rat/static/}},
how it can be used and more details on how it is implemented.

\subsubsection*{Communication with visual editor}

Communication between the visual editor and the GRACe layer takes
place through a RESTful Web service written in Haskell.
%
JSON objects are sent between the two layers via requests to this
service and handled on both ends.

The GRACe web service is available as the \texttt{RestAPI} executable in the
\href{https://github.com/GRACeFUL-project/GRACe}{GRACe} repository.

\todo{Say something about session check stuff.}
\subsubsection*{GCM component libraries}

The visual interface allows the user to access a chosen library of GCM
components.
%
These components are written in GRACe, and each component has a
corresponding JSON interface which is sent to the visual editor when
the user requests the library in question.

\todo{Refer to section on using type systems for testing/verification (and mention our
typed libraries as an example in that section.)}
\subsubsection*{GRACe}

GRACe is a domain specific language embedded in Haskell.
%
It is used to express GRACeFUL concept maps (GCMs) and GCM library
components.
%
GRACe programs representing GCMs are compiled to haskelzinc constraint
programs, and the resulting solutions are passed back to GRACe.

\todo{Refer to GCMP section}
\subsubsection*{Haskelzinc}

Haskelzinc is a Haskell interface to the MiniZinc constraint
programming language.
%
It provides a Haskell DSL for building MiniZinc model representations and a parser that returns a representation of the solutions obtained
by running the MiniZinc model.

The latest version of haskelzinc (currently 0.3) is available from
\url{https://hackage.haskell.org/package/haskelzinc}.

\todo{Refer to section on inductive testing/backend comparison/ satisfiability testing}