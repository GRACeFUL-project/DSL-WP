The GRACeFUL project can be described as a stack of four layers,
starting with stakeholder interaction (group model building, WP2) at
the top, then the graphical user interface (GRACeFUL editor, WP3), the
domain specific language for models and constraints (GRACe DSL, WP4)
and finally the constraint programming (CP) backend providing
efficient solution algorithms (WP5).
%
In addition to the GRACe DSL, WP4 has developed a server around our DSL, called
GRACeServer, for handling webservice requests. The server communicates both
``upwards'' to the GRACeFUL editor and ``downwards'' to the CP layer.
%
This deliverable describes our methods for verifying correctness of
this communication.

The division into frontend, server and backend is very common in
software systems and this means that our results should be applicable
for a wide range of other systems.
%
Our approach to software verification is based on three main parts:
%
declarative programming with strong types, property based testing in
general, and the SessionCheck tool for testing communicating systems
in particular.

\subsection{Scope and limitations}

This deliverable (D4.4) covers the software technology side of testing
and verification of RATs.
%
The CRUD RAT evaluation based on stakeholder sessions is reported
elsewhere [D2.6 Evaluation Report CRUD RATs: m36] as is the
interactivity and usability of the visual front-end [T3.4, D3.3 VA EDA
Tool Prototype (RAT components): m34].

\subsection{Rapid Assessment Tools}

Rapid assessment tools are used in large organisations like the World
Bank and the United Nations to assess risks and needs and to make
plans for improvement.
% http://www.wahooas.org/mshdvd2/assess_tools_MWL_Eng.htm
% http://www.fao.org/docrep/015/i2495e/i2495e06.pdf
In GRACeFUL the main focus is on Climate Resiliant Urban Design, but
the software developed in the project could be used for almost any
Global Systems Science problem.


\subsection{GRACeFUL software architecture}

The software stack of the GRACeFUL project consists of
%
a visual editor frontend,
%
a network layer,
%
a GCM component library,
%
a DSL called GRACe,
%
a middleware called Haskelzinc, and
%
a choice of external constraint solver.
%
In this section we briefly describe the different layers to explain
the context.

\subsection{Visual Editor}

The top layer of the software stack is the visual editor.
%
It provides a graphical user interface where the user can build GCMs
as a graphical map from available components.
%
The visual editor is implemented in the untyped functional language
JavaScript, using the Data Driven Documents library (D3.js).
%
%**TODO perhaps link to the 2017-02 design: https://github.com/GRACeFUL-project/VisualEditor
%**TODO perhaps link to the 2017-05 design (if migration is ready): https://github.com/GRACeFUL-project/GRACeFULEditor

\subsection{Communication with visual editor}

Communication between the visual editor and the GRACe layer takes
place through a RESTful Web service written in Haskell.
%
JSON objects are sent between the two layers via requests to this
service and handled on both ends.

\subsection{GCM component libraries}

The visual interface allows the user to access a chosen library of GCM
components.
%
These components are written in GRACe, and each component has a
corresponding JSON interface which is sent to the visual editor when
the user requests the library in question.

\subsection{GRACe}

GRACe is a domain specific language embedded in Haskell.
%
It is used to express GRACeFUL concept maps (GCMs) and GCM library
components.
%
GRACe programs representing GCMs are compiled to haskelzinc constraint
programs, and the resulting solutions are passed back to GRACe.

\subsection{Haskelzinc}

Haskelzinc is a Haskell interface to the MiniZinc constraint
programming language.
%
It provides
\begin{itemize}
\item a Haskell abstract syntax tree for the MiniZinc language, with
  which one can represent MiniZinc models in Haskell
\item a human-friendly DSL for building MiniZinc model representations
\item a pretty printer to print the representation of a MiniZinc model
  in MiniZinc
\item a parser that returns a representation of the solutions obtained
  by running the MiniZinc model
\item a set of functions useful for building a custom FlatZinc
  solutions parser.
  %
  An additional module gives the possibility to directly get the
  solutions of a MiniZinc finite domain model.
  %**TODO explain what that means
%   Option for interactive interface is provided, as well as choice
%   between two solvers: the G12/FD built-in solver of FlatZinc and
%   choco3.
\end{itemize}

The latest version of haskelzinc (currently 0.3) is available from
\url{https://hackage.haskell.org/package/haskelzinc}.

\subsection{MiniZinc}

MiniZinc is a constraint programming language in which constraint satisfaction
and optimization problems can be modeled independently of constraint solvers.

MiniZinc models are compiled into FlatZinc, a low-level solver input language.
There are a variety of solvers that can be used to solve problems stated in
FlatZinc.

Information on how to install and use MiniZinc is available at
\url{www.minizinc.org}.

