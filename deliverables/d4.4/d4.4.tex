\documentclass{article}
\usepackage[T1]{fontenc}
\usepackage{lmodern}
\usepackage{amssymb,amsmath}
\usepackage{ifxetex,ifluatex}
\usepackage{fixltx2e} % provides \textsubscript
\usepackage[utf8]{inputenc}
\usepackage{color}
\usepackage[unicode=true]{hyperref}
\hypersetup{breaklinks=true,
            bookmarks=true,
            pdfauthor={Patrik Jansson et al.},
            pdftitle={GRACeFUL D4.4: Testing and verification framework},
            colorlinks=true,
            citecolor=blue,
            urlcolor=blue,
            linkcolor=magenta,
            pdfborder={0 0 0}}
\urlstyle{same}  % don't use monospace font for urls
\setlength{\parindent}{0pt}
\setlength{\parskip}{4pt plus 2pt minus 1pt}
\setlength{\emergencystretch}{3em}  % prevent overfull lines

%% Cezar
\usepackage[margin=1.60in]{geometry}
\usepackage[verbose]{wrapfig}
\usepackage{graphicx}
\usepackage{subfig}
\usepackage{rotating}
\usepackage{lscape}
\usepackage{float}
\usepackage{geometry}
\usepackage{framed}
\usepackage{xspace}
\usepackage{acronym}
\usepackage[square,numbers]{natbib}

%% Max
\usepackage{minted}
\usepackage{todonotes}

%% Alex
\newminted{haskell}{fontsize=\normalsize,xleftmargin=2mm,mathescape,linenos}
\newcommand{\haskell}[1]{\mintinline{haskell}|#1|}
\DefineShortVerb{\|}
\AtBeginEnvironment{minted}{%
  \renewcommand{\fcolorbox}[4][]{#4}}

\definecolor{GRACeFULblue}{rgb}{0.20,0.60,0.86}

\newcommand{\grace}{GRACeFUL\xspace}
\acrodef{GCM}{\grace Concept Map}
\acrodef{GMB}{Group Model Building}
\acrodef{DSL}{Domain Specific Language}
\acrodef{CFP}{Constraint Functional Program}
\acrodef{RAT}{Rapid Assessment Tool}
\acrodef{CRUD}{Climate Resilient Urban Design}
\acrodef{CLD}{Causal Loop Diagram}
\acrodef{JSON}{JavaScript Object Notation}
\acrodef{CP}{Constraint Program}
\hyphenation{GRACeFUL}

\author{}
\date{}

\begin{document}

\begin{center}
\includegraphics[width=5cm]{../coverpage/GRACeFULlogo.png}

\textcolor{GRACeFULblue}{Global systems Rapid Assessment tools\\
through Constraint FUnctional Languages}

\vspace{1cm}

FETPROACT-1-2014 Grant Nº 640954

\end{center}

\begin{framed}
\begin{center}
\Large
D4.4: Testing and verification framework\\[1ex]

\large
D4.4 Testing and verification framework for RATs\\
with applications to the CRUD case study\\[1ex]

\end{center}
\end{framed}

\vspace{1cm}

\noindent
\begin{tabular}{@{}ll@{}}
  Lead Participant:       & Chalmers (WP leader: P. Jansson)
\\Dissemination Level:    & PU
\\Document Version:       & Draft
\\Date of Submission:     & 201?-??-??
\\Due Date of Delivery:   & 2018-01-31
\end{tabular}

\newpage

\section*{D4.4 Testing and verification framework for RATs with
  applications to the CRUD case study}

Contributions by: Maximilian Algehed, Sólrún Einarsdóttir, Alex
Gerdes, and Patrik Jansson.

\begin{abstract}

This fourth deliverable (D4.4) of work package 4 presents a framework for
testing and verification of RATs.
%
The work leading up to this deliverable is within Task 4.5 ``Build a
testing and verification framework for RATs'' and the full source code
of the implementation is available on GitHub.


\end{abstract}

\vfill

\setcounter{tocdepth}{2}
\tableofcontents

\vfill


\newpage

\section{Introduction}


\subsection{GRACeFUL software architecture}

The software stack of the GRACeFUL project consists of
%
a visual editor frontend,
%
a network layer,
%
a GCM component library,
%
a DSL called GRACe,
%
a middleware called haskelzinc, and
%
a choice of external constraint solver.
%
In this section we briefly describe the different layers and how testing and
verification is performed on each layer.

\subsubsection*{Visual Editor}

The top layer of the software stack is the visual editor.
%
It provides a graphical user interface where the user can build GCMs
as a graphical map from available components.
%
The visual editor is implemented in the programming language JavaScript, using
the Data Driven Documents library (D3.js).
%

The code for the visual editor can be found in the
\href{https://github.com/GRACeFUL-project}{GRACeFUL-project} GitHub
repository
\href{https://github.com/GRACeFUL-project/GRACeFULEditor}{GRACeFULEditor}.

\todo{Is there anything to say about testing here? Maybe refer to user testing
  related deliverable/stuff}
\todo{AG: mention the evaluation with stakeholders here?}

The visual editor is described in deliverabe D3.3 including how it can be
accessed\footnote{\url{http://vocol.iais.fraunhofer.de/graceful-rat/static/}},
how it can be used, and more details on how it is implemented.

\todo{Refer to section on using type systems for testing/verification (and mention our
typed libraries as an example in that section.)}
\subsubsection*{GRACe}

GRACe is a domain specific language embedded in Haskell. It is used to express
GRACeFUL concept maps (GCMs) and GCM library components. GRACe programs
representing GCMs are compiled to haskelzinc constraint programs, and the
resulting solutions are passed back to GRACe. The DSL is described in deliverable
D4.2.

\subsubsection*{GCM component libraries}

The visual editor allows users to create a GRACeFUL Concept Map with components
of a chosen library. Components are written in GRACe and are grouped in a
library, for example a library with CRUD components. The visual editor can query
the components available in a library using its identifier. A JSON
representation of the components are sent to the visual editor.

The following code shows an example CRUD library, where |rain| and |pump| are
|GCM| components:
\begin{haskellcode}
library :: Library
library = Library "crud"
    [ item "rain" $
        rain ::: "amount" #
          tFloat .-> tGCM          (tPort $ "rainfall" # tFloat)
    , item "pump" $
        pump ::: "capacity" #
          tFloat .-> tGCM (tPair   (tPort $ "inflow"   # tFloat)
                                   (tPort $ "outflow"  # tFloat))
    ]
\end{haskellcode}
The example library makes use of Typed Values, which are explained in
Section~\ref{sec:verification}.

\todo{Say something about session check stuff.}
\subsubsection*{Communication with visual editor}

The GRACeServer provides a RESTful API and uses JSON\footnote{JavaScript Object
Notation} as the exchange format. The server offers the following webservices:
\begin{quote}
\begin{description}
\item [\haskell{libraries}] returns a list with the available component libraries
\item [\haskell{library (id)}] has a library identifier parameter and returns 
  a list with a description (in JSON) of the all library components
\item [\haskell{submit (gcm)}] takes a GRACeFUL Concept Map description as
  argument, which is translated to a GRACe DSL program, and returns the result 
  of the constraint solver
\end{description}
\end{quote}
The visual editor, developed in work package 3, communicates with the GRACeServer
by making service calls. The GRACeServer is available as the \texttt{RestAPI} 
executable in the \href{https://github.com/GRACeFUL-project/GRACe}{GRACe} repository.

\todo{Refer to GCMP section}
\subsubsection*{Haskelzinc}

Haskelzinc is a Haskell interface to the MiniZinc constraint
programming language.
%
It provides a Haskell DSL for building MiniZinc model representations and a parser that returns a representation of the solutions obtained
by running the MiniZinc model.

The latest version of haskelzinc (currently 0.3) is available from
\url{https://hackage.haskell.org/package/haskelzinc}.

\todo{Refer to section on inductive testing/backend comparison/ satisfiability testing}

% Front-end URL: http://vocol.iais.fraunhofer.de/graceful-rat/static/index.html
% TODO (after 2017-11-17 when the front end functionality should be more stable): take screen shot(s) showing some model


\subsection{Scope and limitations}

This deliverable (D4.4) covers the software technology side of testing
and verification of RATs.
%
The CRUD RAT evaluation based on stakeholder sessions is reported
elsewhere [D2.6 Evaluation Report CRUD RATs: m36] as is the
interactivity and usability of the visual front-end [T3.4, D3.3 VA EDA
  Tool Prototype (RAT components): m34].

\subsection{Rapid Assessment Tools}

Rapid assessment tools are used in large organisations like the World
Bank and the United Nations to assess risks and needs and to make
plans for improvement.
% http://www.wahooas.org/mshdvd2/assess_tools_MWL_Eng.htm
% http://www.fao.org/docrep/015/i2495e/i2495e06.pdf
In GRACeFUL the main focus is on Climate Resiliant Urban Design, but
the software developed in the project could be used for almost any
Global Systems Science problem.


\todo{Patrik+Solrun: describe our prototype RAT: screen shot}


\subsection{Property-based testing}
\todo{Alex: introduce property based testing: [before Jan]}
\input{intro-property-based-testing}

\section{}
\todo{Alex: type systems as another testing/verification technique we make use of. TypedValues}

\section{Testing Communicating Systems}
%
SessionCheck \cite{SessionCheck} is a tool developed in WP 4 to test distributed systems like
the GRACeFUL editor.
%
The tool is focused on the communication protocols used between the components.
%
In the case of the GRACeFUL editor the protocol would be the REST API descirbed in \cite{D4.3}. 
%
Typical testing of distrubted systems is done by maintaining a database consisting of hundreds
or thousands of test-cases each specifying the behaviour of one party in the protocol with
respect to a specific exchange of messages, a trace.
%
This approach is not ideal, as each test case is an artifact on its own which needs to be kept
in sync with the rest of the software.
%
SessionCheck releives developers of the duty of maintaining hundreds of software artifacts with
the duty of maintaining three, the client, the server, and the SessionCheck specification.

\begin{figure}
  \begin{minted}{haskell}
    echo = do
      s <- send anyString
      get (is s)
  \end{minted}
  \caption{\label{fig:SessionCheck:echo} The \texttt{echo} protocol}
\end{figure}

See Figure \ref{fig:SessionCheck:echo} for an example of a SessionCheck specfication.
%
The specification specifies the \texttt{echo} protocol, in this protocol the client sends
a single string to the server which replies with the same string.
%
We can use this specification to verify that either a client or a server implementation of
the protocol is correct.
%
For example, when testing a faulty server implementation SessionCheck will print the following:
\begin{verbatim}
TODO TODO TODO TODO
\end{verbatim}

\todo{Maximilian: Testing of communicating systems: keeping components and protocols in sync [draft before Jan]}

\todo{Solrun+Maximilian: Testing solvers and backends: challenges and partial solutions}
% Some notes on things we've found out so far
Sometimes the two backends return different solutions, this seems to be due to
the output minizinc code having a different ordering of variable declaration.
(see https://github.com/GRACeFUL-project/GRACe/tree/master/examples/multi)

When optimizing minizinc/the underlying solver only returns the first optimal solution found, if you ask for more solutions you get intermediate solutions found before the first optimal one.
The way to get all optimal solutions is to get the first optimal one and add it as a constraint for the variable being optimized, then running with solve satisfy and outputting all solutions.
(See pg 29-30 here http://www.minizinc.org/downloads/doc-latest/minizinc-tute.pdf)



\section{Property based testing of GRACe programs}

Maximilian: Tell the story about bugs fixed in the compiler?

\todo{Solrun: try out sessioncheck before Jan.}

\todo{Max: fill in more based on using Jan.}

\section{Conclusion}

\end{document}

From Y2 review: ``We suggest that deliverables consisting of software
products also contain a written document that includes motivations,
use cases, explanations, user’s manuals, and informal semantics (if
possible).''

Note that D4.4 and D2.5 are both deliverables of type

  "DEM: Demonstrator, Pilot, Plans",

which is not described in detail in the H2020 documentation.
