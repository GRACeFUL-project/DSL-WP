\subsubsection{Visual Editor}
The top layer of the software architecture is the visual editor. Written in
JavaScript, it provides a graphical user interface where the user can build
GCMs as a graphical map from available components.

\subsubsection{Communication with visual editor}
Communication between the visual editor and the GRACe layer takes place through
a web service written in Haskell. JSON objects are sent between the two layers
via requests to this service and handled on both ends.  

\subsubsection{GCM component libraries}
The visual interface allows the user to access a chosen library of GCM
components. These components are written in GRACe, and each component has
a corresponding JSON interface which is sent to the visual editor when the
user requests the library in question.

\subsubsection{GRACe}
GRACe is a domain specific language for expressing GRACeFUL concept maps,
embedded in Haskell.

The GRACe program representing a GCM is compiled to a haskelzinc constraint
program.

\subsubsection{Haskelzinc}
Haskelzinc is a Haskell interface to the MiniZinc constraint programming language.
It provides an abstract syntax tree for MiniZinc, with which MiniZinc models can
be represented in Haskell.

\subsubsection{MiniZinc}
Should we say something about MiniZinc?
