\documentclass{article}
\usepackage{todonotes}
\usepackage[T1]{fontenc}
\usepackage{lmodern}
\usepackage{amssymb,amsmath}
\usepackage{ifxetex,ifluatex}
\usepackage{fixltx2e} % provides \textsubscript
\usepackage[utf8]{inputenc}
\usepackage{color}
\usepackage[unicode=true]{hyperref}
\hypersetup{breaklinks=true,
            bookmarks=true,
            pdfauthor={Patrik Jansson et al.},
            pdftitle={GRACeFUL D4.3: Translation of GRACeFUL concept maps to the Constraint Functional Programming layer},
            colorlinks=true,
            citecolor=blue,
            urlcolor=blue,
            linkcolor=magenta,
            pdfborder={0 0 0}}
\urlstyle{same}  % don't use monospace font for urls
\setlength{\parindent}{0pt}
\setlength{\parskip}{4pt plus 2pt minus 1pt}
\setlength{\emergencystretch}{3em}  % prevent overfull lines

%% Cezar
\usepackage[margin=1.60in]{geometry}
\usepackage[verbose]{wrapfig}
\usepackage{graphicx}
\usepackage{subfig}
\usepackage{rotating}
\usepackage{lscape}
\usepackage{float}
\usepackage{geometry}
\usepackage{framed}
\usepackage{xspace}
\usepackage{acronym}
\usepackage[square,numbers]{natbib}

\definecolor{GRACeFULblue}{rgb}{0.20,0.60,0.86}

\newcommand{\grace}{GRACeFUL\xspace}
\acrodef{GCM}{\grace Concept Map}
\acrodef{GMB}{Group Model Building}
\acrodef{DSL}{Domain Specific Language}
\acrodef{CFP}{Constraint Functional Program}
\acrodef{RAT}{Rapid Assessment Tool}
\acrodef{CRUD}{Climate Resilient Urban Design}
\acrodef{CLD}{Causal Loop Diagram}
\hyphenation{GRACeFUL}

\author{}
\date{}

\begin{document}

\begin{center}
\includegraphics[width=5cm]{../coverpage/GRACeFULlogo.png}

\textcolor{GRACeFULblue}{Global systems Rapid Assessment tools\\
through Constraint FUnctional Languages}

\vspace{1cm}

FETPROACT-1-2014 Grant Nº 640954

\end{center}

\begin{framed}
\begin{center}
\Large
Translation of GRACeFUL concept maps\\
to the Constraint Functional Programming layer\\[1ex]

D4.3\\[1ex]

\end{center}
\end{framed}

\vspace{1cm}

\noindent
\begin{tabular}{@{}ll@{}}
  Lead Participant:       & Chalmers (WP leader: P. Jansson)
\\Partners Contributing:  & KU Leuven
\\Dissemination Level:    & PU
\\Document Version:       & Draft
\\Date of Submission:     & 2017-07-07?
\\Due Date of Delivery:   & 2017-07-31
\end{tabular}

\newpage

\section*{Translation of GRACeFUL concept maps to the Constraint Functional Programming layer}\label{DSL4GRACeFUL}

Contributions by: Oskar Abrahamsson, Maximilian Algehed, Sólrún
Einarsdóttir, Alex Gerdes, and Patrik Jansson.

\vfill

\setcounter{tocdepth}{2}
\tableofcontents

\vfill

\newpage

\section*{Abstract}\label{abstract}

This third deliverable (D4.3) of work package 4 presents the
translation of GRACeFUL concept maps (expressed as GRACe programs) to
the Constraint Functional Programming (CFP) layer.
%
This report builds on the description of GRACe in ``D4.2: A Domain
Specific Language (DSL) for GRACeFUL Concept Maps'' (delivered in
project month 24) and the third release of the CFP layer ``haskelzinc''.
%
(The first release was described in ``D5.1: Domain-Specific Language
for the Constraint Functional Programming Platform'' and the latest
version is available from
\url{https://hackage.haskell.org/package/haskelzinc}.)
%

The work leading up to this deliverable is within Task 4.3 ``implement
a middleware for connecting the DSL to the CFP layer'' and he full
source code of the implementation is available on github.


\section{Introduction}


TODO:
\begin{verbatim}
* Introduction explaining the title: What exactly does the title mean?
    * A GRACeFUL concept map is described in D4.2 (basically a GRACe program)
    * The CFP layer is (some version) of the HaskelZinc backend
    * Include (small part of) examples of GRACe code and translation (back and forth)
* Some background about the terminology
\end{verbatim}

\section{A GRACe-ful translation}

\subsection{Examples of GRACe programs and the way they are translated}

\subsection{Software architecture}
The software stack of the GRACeFUL project consists of
%
a visual editor frontend,
%
a network layer,
%
a GCM component library,
%
a DSL called GRACe,
%
a middleware called haskelzinc, and
%
a choice of external constraint solver.
%
In this section we briefly describe the different layers and how testing and
verification is performed on each layer.

\subsubsection*{Visual Editor}

The top layer of the software stack is the visual editor.
%
It provides a graphical user interface where the user can build GCMs
as a graphical map from available components.
%
The visual editor is implemented in the programming language JavaScript, using
the Data Driven Documents library (D3.js).
%

The code for the visual editor can be found in the
\href{https://github.com/GRACeFUL-project}{GRACeFUL-project} GitHub
repository
\href{https://github.com/GRACeFUL-project/GRACeFULEditor}{GRACeFULEditor}.

\todo{Is there anything to say about testing here? Maybe refer to user testing
  related deliverable/stuff}
\todo{AG: mention the evaluation with stakeholders here?}

The visual editor is described in deliverabe D3.3 including how it can be
accessed\footnote{\url{http://vocol.iais.fraunhofer.de/graceful-rat/static/}},
how it can be used, and more details on how it is implemented.

\todo{Refer to section on using type systems for testing/verification (and mention our
typed libraries as an example in that section.)}
\subsubsection*{GRACe}

GRACe is a domain specific language embedded in Haskell. It is used to express
GRACeFUL concept maps (GCMs) and GCM library components. GRACe programs
representing GCMs are compiled to haskelzinc constraint programs, and the
resulting solutions are passed back to GRACe. The DSL is described in deliverable
D4.2.

\subsubsection*{GCM component libraries}

The visual editor allows users to create a GRACeFUL Concept Map with components
of a chosen library. Components are written in GRACe and are grouped in a
library, for example a library with CRUD components. The visual editor can query
the components available in a library using its identifier. A JSON
representation of the components are sent to the visual editor.

The following code shows an example CRUD library, where |rain| and |pump| are
|GCM| components:
\begin{haskellcode}
library :: Library
library = Library "crud"
    [ item "rain" $
        rain ::: "amount" #
          tFloat .-> tGCM          (tPort $ "rainfall" # tFloat)
    , item "pump" $
        pump ::: "capacity" #
          tFloat .-> tGCM (tPair   (tPort $ "inflow"   # tFloat)
                                   (tPort $ "outflow"  # tFloat))
    ]
\end{haskellcode}
The example library makes use of Typed Values, which are explained in
Section~\ref{sec:verification}.

\todo{Say something about session check stuff.}
\subsubsection*{Communication with visual editor}

The GRACeServer provides a RESTful API and uses JSON\footnote{JavaScript Object
Notation} as the exchange format. The server offers the following webservices:
\begin{quote}
\begin{description}
\item [\haskell{libraries}] returns a list with the available component libraries
\item [\haskell{library (id)}] has a library identifier parameter and returns 
  a list with a description (in JSON) of the all library components
\item [\haskell{submit (gcm)}] takes a GRACeFUL Concept Map description as
  argument, which is translated to a GRACe DSL program, and returns the result 
  of the constraint solver
\end{description}
\end{quote}
The visual editor, developed in work package 3, communicates with the GRACeServer
by making service calls. The GRACeServer is available as the \texttt{RestAPI} 
executable in the \href{https://github.com/GRACeFUL-project/GRACe}{GRACe} repository.

\todo{Refer to GCMP section}
\subsubsection*{Haskelzinc}

Haskelzinc is a Haskell interface to the MiniZinc constraint
programming language.
%
It provides a Haskell DSL for building MiniZinc model representations and a parser that returns a representation of the solutions obtained
by running the MiniZinc model.

The latest version of haskelzinc (currently 0.3) is available from
\url{https://hackage.haskell.org/package/haskelzinc}.

\todo{Refer to section on inductive testing/backend comparison/ satisfiability testing}

\subsection{Properties and tests}

\section{Conclusion}

We have designed and implemented a translation from the DSL called
GRACe (describing GRACeFUL concept maps) to the underlying Constraint
Functional Programming layer.
%
We have presented code examples, the software architecture, and
automatically checked properties of the translation.
%
The source code is available on GitHub.

The next actions in work package 4 is
\begin{itemize}
\item build a testing and verification framework for RATs,
\item assist WP3 in implementing the graphical user interface, and
\item assist WP2 in building up a library of GRACeFUL Concept Map
  components expressed in GRACe.
\end{itemize}
%
In parallel, the translation, including the source and target DSLs
will gradually evolve to handle more and more of the requirements
extractable from GMB sessions with stakeholders.


\appendix
\section{Installation and usage}

% ----------------------------------------------------------------

\bibliographystyle{unsrtnat}
\bibliography{d4.3}

\end{document}
