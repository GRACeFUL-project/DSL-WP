\documentclass{article}
\usepackage{todonotes}
\usepackage[T1]{fontenc}
\usepackage{lmodern}
\usepackage{amssymb,amsmath}
\usepackage{ifxetex,ifluatex}
\usepackage{fixltx2e} % provides \textsubscript
\usepackage[utf8]{inputenc}
\usepackage{color}
\usepackage[unicode=true]{hyperref}
\hypersetup{breaklinks=true,
            bookmarks=true,
            pdfauthor={Patrik Jansson et al.},
            pdftitle={GRACeFUL D4.3: Translation of GRACeFUL concept maps to the Constraint Functional Programming layer},
            colorlinks=true,
            citecolor=blue,
            urlcolor=blue,
            linkcolor=magenta,
            pdfborder={0 0 0}}
\urlstyle{same}  % don't use monospace font for urls
\setlength{\parindent}{0pt}
\setlength{\parskip}{4pt plus 2pt minus 1pt}
\setlength{\emergencystretch}{3em}  % prevent overfull lines

%% Cezar
\usepackage[margin=1.60in]{geometry}
\usepackage[verbose]{wrapfig}
\usepackage{graphicx}
\usepackage{subfig}
\usepackage{rotating}
\usepackage{lscape}
\usepackage{float}
\usepackage{geometry}
\usepackage{framed}
\usepackage{xspace}
\usepackage{acronym}
\usepackage[square,numbers]{natbib}

%% Max
\usepackage{minted}
\usepackage{todonotes}
\newminted{haskell}{fontsize=\normalsize,xleftmargin=2mm,mathescape,linenos}
\newcommand{\haskell}[1]{\mintinline{haskell}|#1|}
\DefineShortVerb{\|}
\AtBeginEnvironment{minted}{%
  \renewcommand{\fcolorbox}[4][]{#4}}

\definecolor{GRACeFULblue}{rgb}{0.20,0.60,0.86}

\newcommand{\grace}{GRACeFUL\xspace}
\acrodef{GCM}{\grace Concept Map}
\acrodef{GMB}{Group Model Building}
\acrodef{DSL}{Domain Specific Language}
\acrodef{CFP}{Constraint Functional Program}
\acrodef{RAT}{Rapid Assessment Tool}
\acrodef{CRUD}{Climate Resilient Urban Design}
\acrodef{CLD}{Causal Loop Diagram}
\acrodef{JSON}{JavaScript Object Notation}
\acrodef{CP}{Constraint Program}
\hyphenation{GRACeFUL}

\author{}
\date{}

\begin{document}

\begin{center}
\includegraphics[width=5cm]{../coverpage/GRACeFULlogo.png}

\textcolor{GRACeFULblue}{Global systems Rapid Assessment tools\\
through Constraint FUnctional Languages}

\vspace{1cm}

FETPROACT-1-2014 Grant Nº 640954

\end{center}

\begin{framed}
\begin{center}
\Large
Translation of GRACeFUL concept maps\\
to the Constraint Functional Programming layer\\[1ex]

D4.3\\[1ex]

\end{center}
\end{framed}

\vspace{1cm}

\noindent
\begin{tabular}{@{}ll@{}}
  Lead Participant:       & Chalmers (WP leader: P. Jansson)
\\Partners Contributing:  & KU Leuven
\\Dissemination Level:    & PU
\\Document Version:       & Draft
\\Date of Submission:     & 2017-07-07?
\\Due Date of Delivery:   & 2017-07-31
\end{tabular}

\newpage

\section*{Translation of GRACeFUL concept maps to the Constraint Functional Programming layer}\label{DSL4GRACeFUL}

Contributions by: Oskar Abrahamsson, Maximilian Algehed, Sólrún
Einarsdóttir, Alex Gerdes, and Patrik Jansson.

\vfill

\setcounter{tocdepth}{2}
\tableofcontents

\vfill

\newpage

\section*{Abstract}\label{abstract}

This third deliverable (D4.3) of work package 4 presents the
translation of GRACeFUL concept maps (expressed as GRACe programs) to
the Constraint Functional Programming (CFP) layer.
%
This report builds on the description of GRACe in ``D4.2: A Domain
Specific Language (DSL) for GRACeFUL Concept Maps'' (delivered in
project month 24) and the third release of the CFP layer ``haskelzinc''.
%
(The first release was described in ``D5.1: Domain-Specific Language
for the Constraint Functional Programming Platform'' and the latest
version is available from the Haskell package repository
\href{https://hackage.haskell.org/package/haskelzinc}{Hackage}.)
%
The work leading up to this deliverable is within Task 4.3 ``implement
a middleware for connecting the DSL to the CFP layer'' and the full
source code of the implementation is available on github.


\section{Introduction}

This report describes the third deliverable (D4.3) of work package 4 of the
\grace project. The software we present in this document can be downloaded
from GitHub\footnote{\url{https://github.com/GRACeFUL-project/}}.

The main task of work package 4 is to build a \emph{\ac{DSL}} for \acp{GCM}. A
\ac{GCM} is a representation of policy analysis that contains the main elements
of a policy problem definition, such as goals, criteria, and a description of
the system. It is common practice to simulate a model described by a \ac{GCM},
however, this process is unfortunately both time consuming and expensive. The
\grace project tries to aleviate this problem by expressing a \ac{GCM} as a
\emph{constraint program}, which should reduce the analysis time considerably.

A \ac{GCM} is created by stakeholders in a so-called \ac{GMB} session using a
visual editor. Once the \ac{GCM} is complete, the visual editor submits a
representation\footnote{We use \ac{JSON} as an exchange format} of the \ac{GCM}
to our \ac{DSL} layer, described in deliverable D4.2~\cite{D4.2}, which in turn
passes on the \ac{GCM} to the \ac{CP} layer. The \ac{DSL} can be regarded as an
intermediate layer between the visual editor and the \ac{CP} layer, which
increases modularity, simplifies the translation, and reduces the depency on a
particular constraint solver. 

The main challenge for work package 4 is to create a \ac{DSL} that is expressive
enough to model \acp{GCM} as envisaged by the stakeholders, while still being
able to translate to a constraint program. A constraint program is a
collection of (unknown) variables and constraints, for which a constraint solver
tries to find a solution (values for the unknown variables) that satisfies as
many of those constraints as possible. 

In this document we explain how we translate a program in terms of our \ac{DSL} 
to a constraint program using the \ac{CP} interface provided by work package 5, 
see deliverable~\cite{}. In Section~\ref{sec:translate} we give a short summary
of our \ac{DSL} and show how the main concepts, such as ports and parameters,
are expressed in terms of a constraint program. We continue with an explanation
of the software architecture in Section~\ref{sec:architecture}. We end this 
report with some concluding remarks in Section~\ref{sec:conclusion}.


\section{A GRACe-ful translation}
\label{sec:translate}

This section covers how a model written in our DSL, called GRACe, is translated
to a constraint program. This translation is done in several stages. First we
translate a GRACe program to HaskellZinc~\cite{}, which is developed in work
package 5. In the next stage the program is compiled from the HaskellZinc
interemediate representation to a particular constraint programming language,
which in our case is MiniZinc~\cite{MiniZinc}. The results from the constraint
solver are parsed by our \ac{DSL} and returned to the user. 

We start with a small example to explain the syntax and elements of a GRACe
program. A central element of a GRACe program is a \emph{component}, in fact a
GRACe program is a collection of components that can be connected to each other.
The general structure of a GRACe component starts with the declaration of its
ports and parameters, followed by constraints on those, and ends with exposing
the ports and parameters to other components. The following example defines a
component, which represents a pump with a given capacity:
\begin{haskellcode}
pump :: Int -> GCM (Port Int, Port Int, Param Int)
pump cap = do
  inPort   <- createPort
  outPort  <- createPort
  capParam <- createParam capacity

  constrain $ do
    inflow   <- value inPort
    outflow  <- value outPort
    capacity <- value cap
  
    assert $ inflow `inRange` (0, capacity)
    assert $ inflow === outflow
  
  return (inPort, outPort, capParam) 
\end{haskellcode}
The pump component is implemented as a Haskell function that takes an integer
paramter (|cap|) and returns a GRACe component with two ports, representing the
inflow and outflow of the pump, and a parameter for limiting the pump's
capacity. The type signature on first line in the above code snippet reflects
this. The ports and parameter can be connected to other components, such that
they can interact with other components in a model. A port is created with a
call to the |createPort| function, and is of a particular type, such as the
integer port |flow| defined above. A GRACe parameter is created with the
|createParam| function and can be regarded as a port with an initial value. It
is important to distinguish between normal (Haskell) parameters, such as |cap|,
and parameters defined in the GRACe DSL, such as |capacity|. Parameters defined
in the GRACe DSL can be connected to other components, which may influence its
value. 

The ports and parameters are the interface of a component and we use them to
interact with other components. An important feature of our DSL is that we can
put \emph{constraints} on those ports and parameters. We use these constraints
to model the system dynamics of a GRACeFUL Concept Map. The pump component, for
example, constrains the flow through the pump to be positive and smaller than
its maximum capacity, and ensures that the inflow is equal to the outflow. Using
the |constrain| function we can embed constraints in the component. In fact, the
|constrain| function takes a representation of a constraint program as argument.
A constraint program can query the value of ports and parameters using the
|value| function. These values can then be used to express constraints using the
|assert| function. The |assert| function support many different constraint
expressions.

We can group components, such as the |pump| component, in a library. Using such
a library we can create GRACe DSL programs, which we can analyse with a
constraint solver. Before we continue and define an example GRACe program, let
us first create a second component. The following simple component just exposes
a port with a constant volume of rainfall:
\begin{haskellcode}
rainfall :: Int -> GCM (Port Int)
rainfall volume = do
  volumePort <- createPort
  set volumePort volume
  return volumePort
\end{haskellcode}
The |set| primitive is used to constrain the value of a port to a specific value,
in this case to the |volume| parameter. 

Using the |pump| and |rainfall| components we can create a, somewhat contrived,
\ac{GCM} program:
\begin{haskellcode}
main :: IO ()
main = putStr =<< runGCM prog
  where
    prog :: GCM ()
    prog = do
      (inflow, outflow, _) <- pump 100  -- create a pump
      rain <- rainfall 10               -- let it rain
      link rain inflow                  -- connect rain to pump
      output outflow "pump outflow"
\end{haskellcode}
The above example creates a DSL program called |prog|, which instantiates two
components: a pump with a particular capacity, and some rainfall. The |link| DSL
function on line 8 takes two arguments, which can be either a port or a parameter, and
connects them to each other. We explain the semantics of this connection later in
this section. The |output| primitive can be used to add the value
of a port (or parameter) to the output returned by the constraint solver. The
solver is run by calling the |runGCM| function and returns a string as result,
which we print on the command line.

Before running the constraint solver, the DSL program is translated to a 
corresponding MiniZinc program. The MiniZinc representation is saved in a 
temporary file, which is given to the MiniZinc constraint solver. The example 
above is translated to the following MiniZinc code:
%
\begin{minted}[linenos]{prolog}
  var -10000000..10000000: v3;
  var -10000000..10000000: v2;
  var -10000000..10000000: v1;
  var -10000000..10000000: v0;

  constraint ((v2) == (100));
  constraint (((0) <= (v0)) /\ ((v0) <= (v2)));
  constraint ((v0) == (v1));
  constraint ((v3) == (10));
  constraint ((v3) == (v0));

  solve satisfy;

  output ["{\"pump outflow\" : \(v1)}"];
\end{minted}
%
The generated MiniZinc constraint program consists roughly of four parts: some
variable declarations, constraints on those variables, a strategy how to solve
the constraints, and the desired output. 

MiniZinc variables are declared with the keyword |var| followed by its domain
(in this case a range form -10000000 to 10000000) and then an unique name. These
variables are generated during the translation from a DSL program to a
constraint program. All ports and parameters related to a particular constraint
variable. For example, the |capacity| parameter of the |pump| component is
mapped to the |v2| constraint variable. In the example program we instantiated
the |pump| component with a capacity of 100 units, which is used to initialise the
|capacity| parameter. This initialisation gets translated to a constraint that
equals the value of the constraint variable of corresponding to the |capacity| 
parameter to the given argument (100), as we can see on line 6 in the generated
MiniZinc program.

We can restrict the values of ports and parameters by embedding constrains in a
component. For example, we constrained the value of the |inflow| port of the
|pump| component to be in the range from 0 to the maximum capacity. These
embedded constraints are translated to constraints in the generated MiniZinc
program. On line 7 in the generated MiniZinc program we have a constraint that
limits the value of |v0|, which is the corresponding constraint variable to the
|inPort| port of the |pump| component, to be in the given range.


When we link two ports (or parameters), such as on line 8 in the
example |prog| DSL program, we introduce a constraint in the generated MiniZinc
program that states that the values of the corresponding constraint values
are equal to each other. For example, we linked the rainfall to the input of
the |pump| component, which got translated to the constraint on line 10 in the
MiniZinc program. Note that the generated constraints have some extra pairs of
parentheses to make sure we generate valid programs with the correct precedence,
some of these are superfluous and we may leave them out in the future.

We instruct the constraint solver to search for a solution that fulfils all such
constraints with the |solve satisfy| instruction. 

\todo{AG: explain output}



\section{Software architecture}
\label{sec:architecture}

The software stack of the GRACeFUL project consists of
%
a visual editor frontend,
%
a network layer,
%
a GCM component library,
%
a DSL called GRACe,
%
a middleware called haskelzinc, and
%
a choice of external constraint solver.
%
In this section we briefly describe the different layers and how testing and
verification is performed on each layer.

\subsubsection*{Visual Editor}

The top layer of the software stack is the visual editor.
%
It provides a graphical user interface where the user can build GCMs
as a graphical map from available components.
%
The visual editor is implemented in the programming language JavaScript, using
the Data Driven Documents library (D3.js).
%

The code for the visual editor can be found in the
\href{https://github.com/GRACeFUL-project}{GRACeFUL-project} GitHub
repository
\href{https://github.com/GRACeFUL-project/GRACeFULEditor}{GRACeFULEditor}.

\todo{Is there anything to say about testing here? Maybe refer to user testing
  related deliverable/stuff}
\todo{AG: mention the evaluation with stakeholders here?}

The visual editor is described in deliverabe D3.3 including how it can be
accessed\footnote{\url{http://vocol.iais.fraunhofer.de/graceful-rat/static/}},
how it can be used, and more details on how it is implemented.

\todo{Refer to section on using type systems for testing/verification (and mention our
typed libraries as an example in that section.)}
\subsubsection*{GRACe}

GRACe is a domain specific language embedded in Haskell. It is used to express
GRACeFUL concept maps (GCMs) and GCM library components. GRACe programs
representing GCMs are compiled to haskelzinc constraint programs, and the
resulting solutions are passed back to GRACe. The DSL is described in deliverable
D4.2.

\subsubsection*{GCM component libraries}

The visual editor allows users to create a GRACeFUL Concept Map with components
of a chosen library. Components are written in GRACe and are grouped in a
library, for example a library with CRUD components. The visual editor can query
the components available in a library using its identifier. A JSON
representation of the components are sent to the visual editor.

The following code shows an example CRUD library, where |rain| and |pump| are
|GCM| components:
\begin{haskellcode}
library :: Library
library = Library "crud"
    [ item "rain" $
        rain ::: "amount" #
          tFloat .-> tGCM          (tPort $ "rainfall" # tFloat)
    , item "pump" $
        pump ::: "capacity" #
          tFloat .-> tGCM (tPair   (tPort $ "inflow"   # tFloat)
                                   (tPort $ "outflow"  # tFloat))
    ]
\end{haskellcode}
The example library makes use of Typed Values, which are explained in
Section~\ref{sec:verification}.

\todo{Say something about session check stuff.}
\subsubsection*{Communication with visual editor}

The GRACeServer provides a RESTful API and uses JSON\footnote{JavaScript Object
Notation} as the exchange format. The server offers the following webservices:
\begin{quote}
\begin{description}
\item [\haskell{libraries}] returns a list with the available component libraries
\item [\haskell{library (id)}] has a library identifier parameter and returns 
  a list with a description (in JSON) of the all library components
\item [\haskell{submit (gcm)}] takes a GRACeFUL Concept Map description as
  argument, which is translated to a GRACe DSL program, and returns the result 
  of the constraint solver
\end{description}
\end{quote}
The visual editor, developed in work package 3, communicates with the GRACeServer
by making service calls. The GRACeServer is available as the \texttt{RestAPI} 
executable in the \href{https://github.com/GRACeFUL-project/GRACe}{GRACe} repository.

\todo{Refer to GCMP section}
\subsubsection*{Haskelzinc}

Haskelzinc is a Haskell interface to the MiniZinc constraint
programming language.
%
It provides a Haskell DSL for building MiniZinc model representations and a parser that returns a representation of the solutions obtained
by running the MiniZinc model.

The latest version of haskelzinc (currently 0.3) is available from
\url{https://hackage.haskell.org/package/haskelzinc}.

\todo{Refer to section on inductive testing/backend comparison/ satisfiability testing}



\section{Properties and tests}

As a first step towards a testing and verification framework for
GRACeFUL Rapid Assessment Tools we have instrumented the GRACe
implementation with properties and tests.
%

\todo{explain QuickCheck, Fill in properties and tests, next steps}

\section{Conclusion}
\label{sec:conclusion}

We have designed and implemented a translation from the DSL called
GRACe (describing GRACeFUL concept maps) to the underlying Constraint
Functional Programming layer.
%
We have presented code examples, the software architecture, and
automatically checked properties of the translation.
%
The source code is available on GitHub.

The next actions in work package 4 is
\begin{itemize}
\item build a testing and verification framework for RATs,
\item assist WP3 in implementing the graphical user interface, and
\item assist WP2 in building up a library of GRACeFUL Concept Map
  components expressed in GRACe.
\end{itemize}
%
In parallel, the translation, including the source and target DSLs
will gradually evolve to handle more and more of the requirements
extractable from GMB sessions with stakeholders.

% ----------------------------------------------------------------
\appendix

%% -- APPENDIX: Installation instructions -------------------------------------
%% ----------------------------------------------------------------------------

\section{Installation and usage}
\label{appendix-install}
%
In this section we outline the process of installing the GRACe software and its
required dependencies. We start with an overview of the software dependencies
required for developing GRACe programs (Appendix~\ref{install-overview}).
Following this, we provide installation instructions for a platform-independent
package built on the \href{https://www.docker.com/}{Docker} platform, intended
for users who only wish to execute a pre-existing example
(Appendix~\ref{install-docker}).

%% -- A: Software dependencies ------------------------------------------------

\subsection{Software dependencies}
\label{install-overview}
%
The GRACe language is an embedded domain-specific language implemented in the
\href{https://www.haskell.org/}{Haskell} programming language, and uses the
solver tools from the \href{http://www.minizinc.org/}{MiniZinc} software
distribution. Hence, development and execution of GRACe programs requires the
following software dependencies to be met:
%
\begin{itemize}
  \item[(i)] The MiniZinc and Gecode solver software.
  \item[(ii)] A Haskell toolchain able to download packages from
    \href{https://hackage.haskell.org/}{Hackage}, for instance
    \href{https://www.haskell.org/platform/}{the Haskell Platform}.
\end{itemize}

Detailed instructions for installing the Haskell Platform is available at
\url{https://www.haskell.org/platform/}. The preferred way of installing the
MiniZinc and Gecode components is by way of the bundled binary packages
available at \url{http://www.minizinc.org/software.html}.

Finally, instructions for setting up the GRACe library for building and
executing GRACe programs is available at the \href{https://github.com/GRACeFUL-%
project/GRACe/blob/master/Readme.md}{GRACe GitHub repository}.

%% -- A: Using Docker ---------------------------------------------------------

\subsection{Installation using Docker}
\label{install-docker}

In addition to the installation instructions for the GRACe development tools
(Appendix~\ref{install-overview}) we also provide a platform-independent Docker
image containing an executable for the \verb!OilCrops! example, written in
GRACe.

The \verb!OilCrops! example contains a small optimization problem in which the
objective is to dedicate a set amount of farmland area to three different crops,
with the goal of maximizing the yield of vegetable oil produced from these 
crops. A full description of the example can be found at \url{url.here}. 
\todo{Link to OilCrops.hs `explanation'.}
The GRACe source code of the example can be found at \url{https://github.com/%
GRACeFUL-project/GRACe/blob/master/examples/OilCrops.hs}.

Running the \verb!OilCrops! example requires the Docker Community Edition (CE)
to be installed. Docker CE, as well as installation instructions are available
at \url{https://www.docker.com/products/docker}. Once Docker CE is installed,
the example can be executed using the Docker application as follows. Open a
terminal (the command prompt for Windows users) and execute the commands
%
\begin{verbatim}
  docker pull eugraceful/grace-examples:latest
  docker run --rm eugraceful/grace-examples:latest
\end{verbatim}

\noindent
This will run the \verb!OilCrops! example and write the problem solution to
standard output.

% ----------------------------------------------------------------

\bibliographystyle{unsrtnat}
\bibliography{d4.3}

\end{document}
