%% -- Installation instructions -----------------------------------------------
%% ----------------------------------------------------------------------------

\todo{What exactly do we want to provide installation/usage instructions for?
Should we include the front-end, the web service?}

The purpose of this section is to outline the process of installing
the GRACe software and its required dependencies.
%
Section~\ref{install-overview} provides an overview of the software
dependencies required for development using GRACe.
%
Section~\ref{install-docker} provides installation instructions for a
platform-independent package built on the
\href{https://www.docker.com/}{Docker} platform.

%% -- Software dependencies ---------------------------------------------------

\subsection{Software dependencies}
\label{install-overview}

The GRACe language is implemented in
\href{https://www.haskell.org/}{Haskell} and uses the solver tools
from the \href{http://www.minizinc.org/}{MiniZinc} software
distribution.
%
Specifically, development and execution of GRACe programs requires the
following software dependencies to be met:

\begin{itemize}
\item The MiniZinc and Gecode solver software.
%
  These can be found in the MiniZinc software bundle.
%
\item A complete Haskell toolchain able to download packages from
  \href{https://hackage.haskell.org/}{Hackage}.
%
  Two alternative tools (Stack and Cabal) are suitable for this
  purpose, and both are provided by
  \href{https://www.haskell.org/platform/}{the Haskell Platform}.
\end{itemize}

Instructions for installing and using these components are available
in the {GRACe} repository on
\href{https://github.com/GRACeFUL-project/%
  GRACe/blob/master/doc/INSTALL.md}{GitHub}.

\todo{update the instructions on Github}
%
Alternatively, for those who only wish to run the GRACe
examples, we provide instructions in the following section.

%% -- Using Docker ------------------------------------------------------------

\subsection{Installation using Docker}
\label{install-docker}

Download and install the Docker app from \url{https://www.docker.com/products/docker}.
%

\todo{Create a new docker image, update addresses here.}

Open a terminal (or the \emph{command prompt} under Windows) and
execute
\begin{verbatim}
  docker pull eugraceful/grace-examples
  docker run --rm eugraceful/grace-examples
\end{verbatim}

This will download and execute the example located at
\href{https://github.com/GRACeFUL-project/GRACe/blob/master/examples/Examples.hs}{examples/Examples.hs}.
