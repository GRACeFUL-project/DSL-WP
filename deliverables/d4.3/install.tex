%% -- APPENDIX: Installation instructions -------------------------------------
%% ----------------------------------------------------------------------------

\section{Installation and usage}
\label{appendix-install}
%
In this section we outline the process of installing the GRACe software and its
required dependencies. We start with an overview of the software dependencies
required for developing GRACe programs (Appendix~\ref{install-overview}).
Following this, we provide installation instructions for a platform-independent
package built on the \href{https://www.docker.com/}{Docker} platform, intended
for users who only wish to execute a pre-existing example
(Appendix~\ref{install-docker}).

%% -- A: Software dependencies ------------------------------------------------

\subsection{Software dependencies}
\label{install-overview}
%
The GRACe language is an embedded domain-specific language implemented in the
\href{https://www.haskell.org/}{Haskell} programming language, and uses the
solver tools from the \href{http://www.minizinc.org/}{MiniZinc} software
distribution. Hence, development and execution of GRACe programs requires the
following software dependencies to be met:
%
\begin{itemize}
  \item[(i)] The MiniZinc and Gecode solver software.
  \item[(ii)] A Haskell toolchain able to download packages from
    \href{https://hackage.haskell.org/}{Hackage}, for instance
    \href{https://www.haskell.org/platform/}{the Haskell Platform}.
\end{itemize}

Detailed instructions for installing the Haskell Platform is available at
\url{https://www.haskell.org/platform/}. The preferred way of installing the
MiniZinc and Gecode components is by way of the bundled binary packages
available at \url{http://www.minizinc.org/software.html}.

Finally, instructions for setting up the GRACe library for building and
executing GRACe programs is available at the \href{https://github.com/GRACeFUL-%
project/GRACe/blob/master/Readme.md}{GRACe GitHub repository}.

%% -- A: Using Docker ---------------------------------------------------------

\subsection{Installation using Docker}
\label{install-docker}

In addition to the installation instructions for the GRACe development tools
(Appendix~\ref{install-overview}) we also provide a platform-independent Docker
image containing an executable for the \verb!OilCrops! example, written in
GRACe.

The \verb!OilCrops! example contains a small optimization problem in which the
objective is to dedicate a set amount of farmland area to three different crops,
with the goal of maximizing the yield of vegetable oil produced from these 
crops. A full description of the example can be found at \url{url.here}. 
\todo{Link to OilCrops.hs `explanation'.}
The GRACe source code of the example can be found at \url{https://github.com/%
GRACeFUL-project/GRACe/blob/master/examples/OilCrops.hs}.

Running the \verb!OilCrops! example requires the Docker Community Edition (CE)
to be installed. Docker CE, as well as installation instructions are available
at \url{https://www.docker.com/products/docker}. Once Docker CE is installed,
the example can be executed using the Docker application as follows. Open a
terminal (the command prompt for Windows users) and execute the commands
%
\begin{verbatim}
  docker pull eugraceful/grace-examples:latest
  docker run --rm eugraceful/grace-examples:latest
\end{verbatim}

\noindent
This will run the \verb!OilCrops! example and write the problem solution to
standard output.
