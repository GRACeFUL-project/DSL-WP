\section{Introduction}\label{introduction}

This document reports on the second deliverable (D4.2) of work package 4 of the
\grace project. The main task of this work package is to build a \emph{\ac{DSL}}
for \acp{GCM}. A \ac{GCM} is a representation of policy analysis that contains
all the elements of a policy problem definition, such as goals, criteria, and
  a description of the system. A \ac{GCM} is developed and manipulated by
stakeholders during \ac{GMB} sessions. Ultimately a \ac{GCM} is translated
(via the \ac{DSL}) to a \ac{CFP}, which is analysed by a constraint solver. The
results of the constraint solver are presented to the stakeholders.

The \ac{DSL} can be regarded as an intermediate layer between the visual
representation of a \acf{GCM} and a corresponding constraint program.
Translating the visual representation directly to a constraint program is
difficult, because it is hard to check if the generated program is correct. A
DSL alleviates this problem and allows us to validate the correctness of a
model. In addition, a \ac{DSL} improves the scalability by abstracting away from
the constraint solver. In the longer term this will lead to a \ac{DSL} aimed at
building scalable \acp{RAT} for collective policy making in Global Systems.

% How have the definitions from D4.1 changed

In the previous deliverable~\cite{D4.1} we have formalised the various elements
of \acp{GCM}. Progressive insight showed us that \acp{CLD} are not adequate to
model the systems the project is envisioning, such as for the \ac{CRUD} case
study. Instead of using \aclp{CLD} we now use \emph{stock-and-flow} diagrams to
model the system dynamics. Stock-and-flow diagrams allow for a more detailed
(semi-)quantitative analysis. We have generelised our DSL such that it can handle
\acl{CLD} as well as these stock-and-flow diagrams.

% We have constructed a DSL, link to github

We have implemented the \acl{DSL} in the functional programming language
Haskell~\cite{haskell98}. Haskell is the so-called host language in which the
\ac{DSL} is embedded. Language features such as algebraic datatypes,
higher-order functions, lazy evaluation, and a rich type system makes Haskell
particulary suitable for defined \acp{DSL}. The details of this language are
explained in Section~\ref{sec:gl}. The actual source code implementing the DSL
is freely available in the project repository on
GitHub\footnote{\url{https://github.com/GRACeFUL-project/}}.

% Explain the structure of the rest of the doc

We continue this document by giving the necessary background information, in
Section~\ref{sec:background}, for making the remaining sections more accessible.
We present the details of the \ac{DSL} for \acp{GCM} including some examples in
Section~\ref{sec:gl}. Section~\ref{install-and-reqs} provides installation- and
usage instructions for the \ac{DSL} implementation, as well as an overview of
the required software dependencies. Finally, we give the formal semantics of
the \ac{DSL} in Section~\ref{sec:semantics}.
