\section{Introduction}\label{introduction}

% explain what we mean with an GCM, restricted version: semi-qualitative version
% of CLDs

% Use a DSL as a kind of intermediate representation

% Not easy to translate from a visual diagram directly to CP

% Overview of Graceful, at least the bit of our concern, place the DSL in context

This document reports on second deliverable (D4.2) of work package 4 of the
\grace project. The main task of this work package is to build a \emph{\ac{DSL}}
for \acp{GCM}. A \ac{GCM} is a representation of policy analysis that contains
all the elements of a policy problem definition, such as goals, criteria, and
a description of the system. A \ac{GCM} is developed and manipulated by
stakeholders during \ac{GMB} sessions. Ultimately a \ac{GCM} is translated
(via the \ac{DSL}) to a \ac{CFP}, which is analysed by a constraint solver. The
results of the constraint solver are presented to the stakeholders.

The \ac{DSL} can be regarded as an intermediate layer between the visual
representation of a \acf{GCM} and a corresponding constraint program.
Translating the visual representation directly to a constraint program is
difficult, because it is hard to check if the generated program is correct. A
DSL alleviates this problem and us to validate the correctness of a model. In
addition, a \ac{DSL} improves the scalability by abstracting away from the
constraint solver. In the longer term this will lead to a \ac{DSL} aimed at
building scalable \acp{RAT} for collective policy making in Global Systems.  

% How have the definitions from D4.1 changed

In the previous deliverable~\cite{d4.1} we have formalised the various elements
of \acp{GCM}. Progressive insight learned us that \acp{CLD} are not
adequate to model the systems the project is envisioning, such as for the \ac{CRUD}
case study. Instead of using \aclp{CLD} we now use
\emph{stock-and-flow} diagram to model the system dynamics. Stock-and-flow
diagrams allow for a more detailed (semi-)quantative analysis.

% We have constructed a DSL, link to github

We have implemented the \acl{DSL} in the functional programming language
Haskell~\cite{haskell98}. Haskell is the so-called host language in which the
\ac{DSL} is embedded. Language features such as algebraic datatypes,
higher-order functions, lazy evaluation, and a rich type system makes Haskell
particulary suitable for defined \acp{DSL}. The details of this language are
explained in Section~\ref{sec:gl}. The actual source code implementing the DSL
is freely available in the project repository on
GitHub\footnote{\url{https://github.com/GRACeFUL-project/}}.

% Explain the structure of the rest of the doc



%The overall purpose of WP4 is to use a DSL for GRACeFUL concept maps,
%logic and relations to bridge between the complexity of the CRUD case
%study from WP2 and the underlying science and technology of WP5.   During the
%project we work with embedded DSLs to improve scalability,
%verifiability and correctness of the models.

%Deliverable D4.1 reported on task T4.1 and this document mainly
%reports on the work in tasks T4.2 and T4.3 but also includes some
%initial results from T4.4 and T4.5.

%\begin{itemize}
%\item T4.1 identify key underlying concepts needed for the CRUD case study
%\item T4.2 develop a DSL to describe the concept maps developed during GMB sessions
%\item T4.3 provide a formal semantics for the elements of the DSL
%\item T4.4 implement a middleware for connecting the DSL to the CFP layer
%\item T4.5 build a testing and verification framework for RATs
%\end{itemize}

