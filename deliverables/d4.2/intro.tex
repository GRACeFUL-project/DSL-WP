\section{Introduction}\label{introduction}

This document reports on second deliverable (D4.2) of work package 4 of the
\grace project. The main task of this work package is to build a \emph{Domain
Specific Language} (DSL) for \grace Concept Maps (GCMs). A GCM is a
representation of policy analysis that contains all the elements of a policy
problem definition, such as goals, criteria, and a description of the system as
a stock-and-flow diagram. A GCM is developed and manipulated by stakeholders
during Group Model Building (GMB) sessions. Ultimately a GCM is translated via a
DSL to a \emph{constraint functional program}, which is analysed by a constraint
solver. The results of the constraint solver is presented to the stakeholders.


The DSL can be regarded as an intermediate layer between the visual representation
of a \grace concept map and a corresponding constraint program. Translating the
visual representation directly to a constraint program is difficult \todo{ref}.
A DSL enables alleviates this problem and 
us to , and abstract away from the constraint solver.

% explain what we mean with an GCM, restricted version: semi-qualitative version
% of CLDs

% Use a DSL as a kind of intermediate representation

% Not easy to translate from a visual diagram directly to CP

% Overview of Graceful, at least the bit of our concern, place the DSL in context

insert graph

% How have the definitions from D4.1 changed

In the previous deliverable we have formalised the different elements of \grace
concept maps. Progressive insight learned us that causal loop diagrams are not
adequate to model the systems we are envi 

which allows for a more detailed (semi-)quantative analysis.



% We have constructed a DSL, link to github

% Explain the structure of the rest of the doc



The overall purpose of WP4 is to use a DSL for GRACeFUL concept maps,
logic and relations to bridge between the complexity of the CRUD case
study from WP2 and the underlying science and technology of WP5.  In
the longer term this will lead to a DSL aimed at building scalable
RATs for collective policy making in Global Systems.  During the
project we work with embedded DSLs to improve scalability,
verifiability and correctness of the models.

Deliverable D4.1 reported on task T4.1 and this document mainly
reports on the work in tasks T4.2 and T4.3 but also includes some
initial results from T4.4 and T4.5.

%\begin{itemize}
%\item T4.1 identify key underlying concepts needed for the CRUD case study
%\item T4.2 develop a DSL to describe the concept maps developed during GMB sessions
%\item T4.3 provide a formal semantics for the elements of the DSL
%\item T4.4 implement a middleware for connecting the DSL to the CFP layer
%\item T4.5 build a testing and verification framework for RATs
%\end{itemize}

