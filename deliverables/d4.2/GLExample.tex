We show a small GL program which models a rain runoff area,
like a town square, which has been provided with a pump to
alleviate possible flooding issues (this is a common procedure
in countries like the Netherlands). This example is a larger
part of the CRUD case study, meant to show how GL can be
employed to model concrete problems in CRUD.

We omit the definition of the
more complicated parts of the example for brevity.
\begin{verbatim}
data Pump = Pump {inflowP, outflowP :: Port Float}

pump :: Float -> GCM Pump
pump capacity = do
  inPort  <- createPort
  outPort <- createPort
  component $ do
    i <- value inPort
    o <- value outPort
    assert $ i `inRange` (0, lit capacity)
    assert $ i === o
  return $ Pump inPort outPort

rain :: Float -> GCM (Port Float)
rain flow = do
  rainP <- createPort
  set rainP flow

data Storage = Storage {inflowS, outflowS, overflowS :: Port Float}

storage :: Float -> GCM Storage
storage = ... -- The implementation of storage is quite involved

example :: GCM ()
example = do
  p <- pump 5
  s <- storage 4
  r <- rain 10

  link (inflowP p) (outflowS s)
  link (inflowS s) r

  output "Overflow" (overflowS s)
  \end{verbatim}

When we run the above program we get the output

\begin{verbatim}
ghci> runGCM example
{"Overflow" : 0}
\end{verbatim}

\todo{Maybe we should have a more elaborate example which utlises actions?}
