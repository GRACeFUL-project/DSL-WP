\subsection{Example: Runoff flow}
\label{example-runoff-flow}

We show a small GL program which models a rain runoff area, like a
town square, which has been provided with a pump to alleviate possible
flooding issues (this is a common procedure in countries like the
Netherlands).
%
This example is a small part of a larger model used in the CRUD case
study meant to show how GL can be employed to model concrete problems
in CRUD.
%
The program has three components and Fig. \ref{fig:RunoffEx} shows a
graphical view of how the components are connected.
%

Given the components defined earlier, \texttt{pump}, \texttt{rain}, and \texttt{runoffArea},
we may express Fig. \ref{fig:RunoffEx} as:
\begin{verbatim}
example :: GCM ()
example = do
  (inflowP, outflowP) <- pump 5
  (inflowS, outletS, overflowS) <- runoffArea 4
  rainflow <- rain 10

  link inflowP outletS
  link inflowS rainflow

  output "Overflow" overflowS
\end{verbatim}
The \texttt{output} command lets us inspect the resulting value at a \texttt{Port}
after all constraints have been solved.

Our example is compiled and run using the \texttt{runGCM} command.
%
The constraint programming runtime, \texttt{MiniZinc}, informs us
that with the numbers in our example the overflow will be zero.

\begin{verbatim}
ghci> runGCM example
{"Overflow" : 0}
\end{verbatim}

While this example is easily calculated by hand GL is capable of
expressing more complicated, and less self-contained, GRACeFUL concept
maps including actions and optimization problems.
%
As these examples are somewhat larger we omit them in this document
and refer to the online resources\footnote{See the examples directory
  of the GitHub repository for
  \href{https://github.com/GRACeFUL-project/GenericLibrary}{GenericLibrary}.}.

% -- A GCM representing a simluation of the swedish energy system
% energySystem :: GCM ()
% SmallExample.hs
