GenericLibrary, henceforth abbreviated to GL, is a DSL for descriptions
of GRACeFUL concept map components in Haskell. The DSL addresses the issue
of bridging the gap between constraint programming and the visualisation
layer by providing abstractions for modular constraint programming.

The DSL is divided in to two parts. The first part, \texttt{GCM}, allows the user to describe the
interactions of GRACeFUL concept map components and has facilities for
constructing new components from existing ones. The second part,
\texttt{CP}, features primitives for constructing constraint programs
which describe the behaviour of an individual components.

\todo{This section needs clarification}
The architecture is demonstrated by the example below. A \texttt{GCM}
component exposing a single port is defined, it contains a small \texttt{CP} program (in the
\texttt{component} block) which describes the local behaviour of the
component and features a local variable which is unreachable from the
outer, structural, description.

\todo{if possible, make this example nicer}
\begin{verbatim}
gcmComponent :: GCM (Port Float)
gcmComponent = do
  externalPort <- createPort
  component $ do
   variable <- createLVar
   assert (externalPort === variable + 1)
  return externalPort
\end{verbatim}
