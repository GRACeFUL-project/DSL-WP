\todo{Make nicer}
GenericLibrary, henceforth abbreviated to GL, is a DSL for descriptions
of GRACeFUL concept map components embedded in the Haskell programming language.
The DSL addresses the issue of bridging the gap between constraint programming
and the visualisation layer by providing abstractions for modular constraint programming.
These abstractions are targeted at simplifying the description of GRACeFUL concept maps.

The DSL is divided in to two parts. The first part, \texttt{GCM} \todo{GCM is a monad},
allows the user to describe the interactions of GRACeFUL concept map components
and has facilities for constructing new components from existing ones \todo{Also known as monadic bind}. The second part,
\texttt{CP} \todo{CP is also a monad}, features primitives for constructing constraint programs
which describe the behaviour of an individual components.

\todo{How do we describe the available language constructs?}

\subsection{The language}\todo{Something about Haskell metaprogramming}
\todo{flesh out?}
As noted earlier, the GL language is primarily constructed from two different languages,
GCM and CP.

\todo{Something about how to include a CP component in a GCM one?}
\todo{\texttt{component :: CP a -> GCM ()}, the \texttt{GCM ()} type ensures nothing directly escapes from the scope of \texttt{CP} to \texttt{GCM}}
The GCM language supports constructing interfaces between components,
with language constructs such as \texttt{createPort} for creating interfaces, and has support for
connecting the interface of different components using the \texttt{link}
as well as \texttt{linkBy} primitives.

The CP language supports reasoning about integer and floating-point arithmetic, boolean expressions,
and arrays. It has constructions like \texttt{createLVar} and \texttt{assert} for reasoning about
the internal behaviour of a GRACeFUL concept map component.

As can be seen there is a clear separation of concerns in GL between the high-level primitives
for constructing complicated components from simpler ones, represented in GCM, and the low level
implementation details with which the CP language is concerned.

\subsection{Factors, Actions, and Goals}
The GL languages features primitives for expressing factors, actions, and goals. However, all these are
generalised by the concept of a \texttt{GCM} component.
