\todo{Make nicer}
GenericLibrary, henceforth abbreviated to GL, is a DSL for descriptions
of GRACeFUL concept map components embedded in the Haskell programming language.
The DSL addresses the issue of bridging the gap between constraint programming
and the visualisation layer by providing abstractions for modular constraint programming.
These abstractions are targeted at simplifying the description of GRACeFUL concept maps.

The DSL is divided in to two parts. The first part, \texttt{GCM} \todo{GCM is a monad},
allows the user to describe the interactions of GRACeFUL concept map components
and has facilities for constructing new components from existing ones \todo{Also known as monadic bind}. The second part,
\texttt{CP} \todo{CP is also a monad}, features primitives for constructing constraint programs
which describe the behaviour of an individual components.

\todo{How do we describe the available language constructs?}

\subsection{The language}\todo{Something about Haskell metaprogramming}
\todo{flesh out?}
As noted earlier, the GL language is primarily constructed from two different languages,
GCM and CP.

\todo{Something about how to include a CP component in a GCM one?}
\todo{\texttt{component :: CP a -> GCM ()}, the \texttt{GCM ()} type ensures nothing directly escapes from the scope of \texttt{CP} to \texttt{GCM}}
The GCM language supports constructing interfaces between components,
and has support for connecting the interface of different components.
The core abstraction in GL is that of the \texttt{Port}, a port is an
entity which represents the way two components interact. A component
exposes some information about the system, a pump may present one port
representing the amount of water being pumped through the pump and another
port representing the maximum flow the pump is able to produce. Ports
may be connected to each other through the \texttt{link} function.
\todo{something about the way in which relationships between factors are represented, i.e. \texttt{linkBy}}

The CP language supports reasoning about integer and floating-point arithmetic, boolean expressions,
and arrays. It has constructions like \texttt{createLVar} and \texttt{assert} for reasoning about
the internal behaviour of a GRACeFUL concept map component.

As can be seen there is a clear separation of concerns in GL between the high-level primitives
for constructing complicated components from simpler ones, represented in GCM, and the low level
implementation details with which the CP language is concerned.

\subsection{Expressing GRACeFUL concept map elements in GL}\todo{Relate to d4.1}
Many of the elements of GRACeFUL concept maps identified in \cite{d4.1} can be modeled in
GL using language primitives such as \texttt{linkBy} (for connections),
\texttt{createAction} (for actions), \texttt{assert} (for constraints) etc.
GL being an \textit{embedded} domain specific language featuring a rich set of primitive
operations for reasoning in the target domain allows the programmer to construct
abstractions which capture behaviour which generalises many of the concepts described
in \cite{d4.1}.
