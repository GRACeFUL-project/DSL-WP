\todo{Make nicer}
GenericLibrary, henceforth abbreviated to GL, is a DSL for descriptions
of GRACeFUL concept map components in Haskell. The DSL addresses the issue
of bridging the gap between constraint programming and the visualisation
layer by providing abstractions for modular constraint programming.

The DSL is divided in to two parts. The first part, \texttt{GCM}, allows the user to describe the
interactions of GRACeFUL concept map components and has facilities for
constructing new components from existing ones. The second part,
\texttt{CP}, features primitives for constructing constraint programs
which describe the behaviour of an individual components.

\todo{This section needs clarification}
The architecture is demonstrated by the example below. A \texttt{GCM}
component exposing a single port is defined, it contains a small \texttt{CP} program (in the
\texttt{component} block) which describes the local behaviour of the
component and features a local variable which is unreachable from the
outer, structural, description.

\todo{if possible, make this example nicer}
\begin{verbatim}
gcmComponent :: GCM (Port Float)
gcmComponent = do
  externalPort <- createPort
  component $ do
   variable <- createLVar
   assert (externalPort === variable + 1)
  return externalPort
\end{verbatim}

\todo{How do we describe the available language constructs?}

\subsection{The language}\todo{Something about Haskell metaprogramming}
\todo{flesh out?}
As noted earlier, the GL language is primarily constructed from two different languages,
GCM and CP.

\todo{Something about how to include a CP component in a GCM one?}
The GCM language supports reasoning about components and the interfaces between components,
with language constructs such as \texttt{createPort} for creating interfaces and \texttt{link}
as well as \texttt{linkBy} for connecting interfaces to each other.

The CP language supports reasoning about integer and floating-point arithmetic, boolean expressions,
and arrays. It has constructions like \texttt{createLVar} and \texttt{assert} for reasoning about
constraint variables.

As can be seen there is a clear separation of concerns in GL between the high-level primitives
for constructing complicated components from simpler ones, represented in GCM, and the low level
implementation details with which the CP language is concerned.
