GenericLibrary, henceforth abbreviated to GL, is a DSL for descriptions
of GRACeFUL concept map components embedded in the Haskell programming language.
The DSL addresses the issue of bridging the gap between constraint programming
and the visualisation layer by providing abstractions for modular constraint programming.
These abstractions are targeted at simplifying the description of GRACeFUL concept maps.

The DSL is divided in to two parts. The first part, \texttt{GCM},
allows the user to describe the interactions of GRACeFUL concept map components
and has facilities for constructing new components from existing ones. The second part,
\texttt{CP}, features primitives for constructing constraint programs
which describe the behaviour of an individual components.

\subsection{The language}
The GCM language supports constructing interfaces between components,
and has support for connecting the interface of different components.
The core abstraction in GL is that of the \texttt{Port}, a port is an
entity which represents the way two components interact. A component
exposes some information about the system, a pump may present one port
representing the amount of water being pumped through the pump and another
port representing the maximum flow the pump is able to produce.

We show a somewhat simpler example, a \texttt{GCM} component modelling
a fixed amount of rain falling from the sky.
\begin{verbatim}
rain :: Float -> GCM (Port Float)
rain amount = do
  port <- createPort
  set port amount
  return port
\end{verbatim}

The CP language supports reasoning about integer and floating-point arithmetic, boolean expressions,
and arrays. It has constructions like \texttt{value}, which reads the value from a port, and \texttt{assert} for
expression constraints on the behaviour of a component. Computations in \texttt{CP} can be
embedded in \texttt{GCM} using the \texttt{component} primitive.

We can now return to our pump, which is a \texttt{GCM} component parameterised over the maximum
flow through the pump
\begin{verbatim}
pump :: Float -> GCM (Port Float, Port Float)
pump maxCap = do
  inPort  <- createPort
  outPort <- createPort
  component $ do -- This is in CP
    inflow  <- value inPort
    outflow <- value outPort
    assert $ inflow === outflow
    assert $ inflow `inRange` (0, lit maxCap)
  return (inPort, outPort)
\end{verbatim}
Note that we need to use \texttt{lit} to lift \texttt{maxCap}, which is a value in Haskell,
in to GL.

As can be seen there is a clear separation of concerns in GL between the high-level primitives
for constructing complicated components from simpler ones, expressed in \texttt{GCM}, and the low level
implementation details with which the \texttt{CP} language is concerned.

\subsection{Expressing GRACeFUL concept map elements in GL}
Many of the elements of GRACeFUL concept maps identified in \cite{D4.1} can be modeled in
GL using language primitives such as \texttt{linkBy} (for connections),
\texttt{createAction} (for actions), \texttt{assert} (for constraints) etc.
GL being an \textit{embedded} domain specific language featuring a rich set of primitive
operations for reasoning in the target domain allows the programmer to construct
abstractions which capture behaviour which generalises many of the concepts described
in \cite{D4.1}.
