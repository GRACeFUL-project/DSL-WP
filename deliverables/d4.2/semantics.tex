\section{Formal semantics}
\label{sec:semantics}
% stock & flow - CLD simplification (specialization/simplification of GCM)
% Why formalize?
We would like to define formal semantics for our DSL in order to be
able to reason formally about it.
%
By formal reasoning we can confirm the DSL's robustness and gain
further insight into it.

We started by considering causal loop diagrams, a specialization of
the GCMs our DSL describes.
%
This was done to simplify the initial scope of the work, with the
expectation that the semantics defined for CLDs could then be extended
to the more general GCMs.
%
Work to extend and generalize these semantics to describe
stock-and-flow diagrams, and thereby make them more consistent with
the current implementation of the DSL, is ongoing.

We have also modelled CLDs \emph{within} the \verb|GenericLibrary|:
the code for this model can be found in the file
\verb|QualitativeExample.hs|.

\subsection{Causal loop diagrams}
%
A causal loop diagram (CLD) is a directed graph used to display causal
relationships between variables.
%
The vertices represent the variables and the edges represent
qualitative causal relationships, which can be positive or negative.

Different approaches can be taken in defining formal semantics to aid
us in reasoning about CLDs.
%
We have considered two such approaches: one based on qualitative
probabilistic networks and the other on difference equations.
%
We describe and compare these two approaches in the following
sections.

\subsubsection{Notation}
We denote a positive causal relationship between $A$ and $B$ by
$A\xrightarrow{+} B$ and a negative one by $A \xrightarrow{-} B$.
%
Then $A \xrightarrow{+} B$ informally means that an increase in $A$
causes an increase in $B$ (and a decrease in $A$ causes a decrease in
$B$).
%
On the other hand, $A\xrightarrow{-} B$ means that an increase in $A$
causes a decrease in $B$ (and conversely a decrease in $A$ causes an
increase in $B$).
%
We denote the sign of the edge from $A$ to $B$ by $s_{AB}$, so
$s_{AB}= +$ if $A\xrightarrow{+} B$ and $s_{AB}=-$ if
$A\xrightarrow{-} B$.

A vertex $A$ also has a sign $s_A$ that denotes the total influence on
$A$, so $s_A=+$ if there is an increase in $A$, $s_A=-$ if there is a
decrease, $s_A=0$ if there is no change and $s_A=?$ if we cannot
determine the change in $A$.

\subsection{Qualitative Probabilistic Networks}
One approach to modelling and reasoning about CLDs is by using
qualitative probabilistic networks (QPNs).

A QPN \cite{Wellman} is defined as a directed acyclic graph $G=(V,E)$
where the vertices, $V$, correspond to variables and the edges, $E$ to
qualitative probabilistic influences.
%
These influences can be positive (+) or negative (-).
%
The signs (?), for ambiguous influence, and (0), for probabilistic
independence, can also be used to describe probabilistic
relationships.

The meaning of signs on edges is defined according to first order
stochastic dominance, as follows:

Let $F_B(\cdot|a_i, x)$ be the cumulative distribution function (CDF)
for $B$ given $A=a_i$.
%
Then $s_{AB}=+$ means that for all possible values $a_1,a_2$ of $A$
where $a_1\geq a_2$, we must have:
%
\[F_B(b_0|a_1, x)\leq F_B(b_0|a_2, x),\]
%
that is,
%
\[P(B \leq b_0| A = a_1, x)\leq P(B\leq b_0| A = a_2, x)\]
%
for all possible values $b_0$ of $B$ and any consistent context $x$.
%
The context $x$ ranges over all possible assignments to the variables
other than $A$ that influence $B$, that are consistent with both
$A=a_1$ and $A=a_2$.
%
The definition of $s_{AB}=-$ is the same but with $a_1\leq a_2$.

In simpler terms, $s_{AB} = +$ means that greater values of $A$ mean
greater values of $B$ are more likely, and $s_{AB}=-$ means that
greater values of $A$ mean smaller values of $B$ are more likely.

These influences are symmetric, that is, if the edge from $X$ to $Y$
is reversed we must have $s_{XY} = s_{YX}$.
%
Due to this symmetry it is possible to propagate an observed increase
or decrease of one variable around the graph and find if other
variables are likely to have increased or decreased.

This definition is broad enough to apply to many different systems and
to be applicable to various real world situations.

\subsubsection{Issues} \label{qpnIssues}

We found some issues with QPNs that lead us to explore other
approaches.

First of all, since QPNs were originally defined for acyclic graphs
and the theory on them relies on acyclicity, they may not be the best
fit to describe CLDs, in which cycles (feedback loops) are an
important feature.
%
Inference on QPNs containing loops is difficult to implement, and can
lead to ambiguous results.

Second, the formal semantics of inference on QPNs is difficult to
formalize since it relies heavily on not-so-simple probability theory.
%
Additionally, QPNs are defined solely based on qualitative
relationships and there is no obvious way to expand them to also
describe quantitative relationships unless we have information about
the probability distributions and conditional probabilities involved.
%
GMB sessions will not produce data on probability distributions and
estimating such probabilities in a reliable manner requires a sizeable
dataset (that may not be available for a given GCM) as well as
statistical expertise.

Lastly, since all inference in QPNs is probabilistic it leads to
results that may not be as meaningful or concrete as we would like,
such as ``there is a heightened probability that $x$ has increased'',
rather than ``$x$ has increased''.
%
For instance, a variable may decrease even though the cumulative
probabilistic influence on it is positive.

\subsection{Difference equation approach}

Inspired by a system of tanks with water flowing from one to another,
and in search of semantics that might also be extended to quantitative
reasoning, we came up with the following approach.

We consider the values of the graph's vertices to be functions of the
same variable, such as a time variable $t$.

If we have a graph with two vertices, $X$ and $Y$, and one edge from
$X$ to $Y$, then $s_{XY}=+$ implies that
%
\[\frac{\partial Y}{\partial t} = G(X(t)),\]
%
where $G$ is a monotonically increasing function (monotonically decreasing for negative
causality, $s_{XY}=-$).

If the vertex $Y$ has multiple parent vertices $X_1,\ldots,X_n$, then
$\frac{\partial Y}{\partial t}$ depends on all the parent vertices.
%
We can isolate the effect of a single parent vertex $X$ on $Y$ by
differentiation.

In general we can then describe the causal relationship from $X$ to
$Y$ as
%
\[\frac{\partial\left( \frac{\partial Y(t)}{\partial t} \right)}{\partial Y(t)} =
  g(X(t)),\]
%
where $g$ has a primitive function $G$ such that $G$ is monotonically
increasing if $s_{XY}=+$ and monotonically decreasing if $s_{XY}= -$.

This is somewhat more nuanced than $CLDs$ as they are described above,
where $s_{XY}=+$ implies that an increase in $X$ leads to an increase
in $Y$, and a decrease in $X$ to a decrease in $Y$.
%
Here we may have some threshold value $x_0$ for $X$, where
$G(x_0) = 0$, above which $X$ always causes an increase in $Y$, but an
increase in $X$ causes a faster rate of increase in $Y$ and a decrease
in $X$ causes a slowed rate of increase in $Y$, and vice versa.

Note that though $G(X)$ is monotonically increasing, it may not be
strictly increasing, so we could for instance have $G(X) = 0$ for all
$X < C$ for some threshold value $C$.

If the vertex $Y$ has parent vertices $X_1,\ldots,X_n$, then we have
\[\frac{\partial Y}{\partial t} = \sum_{i=1}G_i(X_i),\]
where $G_i$ is monotonically increasing if $s_{X_iY}=+$ but monotonically
decreasing if $s_{X_iY}=-$.

In a discrete time system we consider $\Delta(X_t) = X_t - X_{t-1}$
instead of $\frac{\partial X}{\partial t}$, and write
$\Delta(X_t) = G(Y_{t-1})$ instead of
$\frac{\partial X}{\partial t} = G(Y(t))$.
%
In simple cases we may only consider one time step with two values of
$t$: $t_{start}$ and $t_{end}$.

Here we explore how this approach relates to qualitative reasoning,
but it could be extended to quantitative reasoning by solving the
appropriate differential equations.

\subsubsection{Simple qualitative model}

We consider a qualitative discrete time system where all values of
vertex variables are either +, -, 0, or ? (where ? is an ambiguous
value assigned to a variable whose value cannot be deduced).
%
These values have the partial ordering $- < 0 < +$, but ? cannot be
compared to the other values.
%
In place of addition and multiplication we have the operations
$\oplus$ and $\otimes$, whose behaviour can be seen in the following
tables:
\begin{center}
\begin{tabular}{c|cccc}
$\oplus$ & + & - & 0 & ?\\
\hline
  +   & +  & ? & + & ?\\
  -   & ?  & - & - & ?\\
  0   & +  & - & 0 & ?\\
  ?   & ?  & ? & ? & ?\\
\end{tabular}
\quad
\begin{tabular}{c|cccc}
$\otimes$ & + & - & 0 & ?\\
\hline
  +   & +  & - & 0 & ?\\
  -   & -  & + & 0 & ?\\
  0   & 0  & 0 & 0 & 0\\
  ?   & ?  & ? & 0 & ?\\
\end{tabular}
\end{center}
The only strictly increasing function in this system is the identity
function $id(x) = x$, and the only strictly decreasing function is the
negation function $neg(x) =-\otimes x$.

For simplicity we consider the case where all initial values are set
to zero and $G_e(0)=0$, for all edges $e$.
%
We only consider edge functions $G_e$ where $G_e(?) = ?$, since we
shouldn't be able to make unambiguous deductions based on ambiguous
values.
%

This is convenient for qualitative reasoning since then we are only
concerned with increases and decreases rather than numerical values.
%
The value of variable $X$ at time $t$, which we denote by $X_t$, then
tells us whether there has been a net increase or decrease in $X$.

Consider a graph with three vertices, $Z$ and its two parents $X$ and
$Y$, $X\xrightarrow{s_{XZ}} Z$ and $Y\xrightarrow{s_{YZ}} Z$.
%
Then we have
%
\[\Delta(Z_t) = G_{XZ}(X_{t-1}) \oplus G_{YZ}(Y_{t-1}),\]
%
where $G_{XZ}$ and $G_{YZ}$ are monotonically increasing or decreasing
in accordance with $s_{XZ}$ and $s_{YZ}$.
%
If we only allow strictly increasing/decreasing functions we then have
%
\[\Delta(Z_t) = (s_{XZ}\otimes X_{t-1})\oplus (s_{YZ}\otimes Y_{t-1}).\]

Consider a graph with three vertices $A$, $B$ and $C$ and two edges,
$A\xrightarrow{s_{AB}} B$ and $B\xrightarrow{s_{BC}} C$.
%
Then we have
\begin{align*}
\Delta(C_t) &= G_{BC}(B_{t-1})\\
&= G_{BC}(B_{t-2} \oplus \Delta(B_{t-1}))\\
&= G_{BC}(B_{t-2} \oplus G_{AB}(A_{t-2}))\\
\end{align*}

If $G_{BC}$ is linear, as is the case when we restrict the available
functions to the strictly increasing/decreasing $id$ and $neg$, we
then have
%
\[\Delta(C_t) = G_{BC}(B_{t-2})\oplus G_{BC}\circ G_{AB}(A_{t-2}).\]

If we only allow strictly increasing/decreasing functions we then have
%
\[\Delta(C_t) = (s_{BC}\otimes B_{t-2})\oplus (s_{BC}\otimes
  s_{AB}\otimes A_{t-2}),\]

\subsection{Comparison of approaches}

We achieve the same results when inferring on CLDs no matter whether
we use the QPN approach or the difference equation approach to
describe the underlying semantics.
%
Which method is simpler to understand and reason about is a matter of
opinion, but we encountered some difficulties when working with the
QPN approach that are outlined in section \ref{qpnIssues}, which make
us sceptical of the extensibility of that approach to quantitative
analysis.

We believe the difference equation method is well suited to describing
stock-and-flow diagrams as the approach was originally inspired by
considering stock-and-flow systems, and we are working towards
extending from CLDs to the more general GCMs.
