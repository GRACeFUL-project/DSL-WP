\documentclass[]{article}
\usepackage{todonotes}
\usepackage[T1]{fontenc}
\usepackage{lmodern}
\usepackage{amssymb,amsmath}
\usepackage{ifxetex,ifluatex}
\usepackage{fixltx2e} % provides \textsubscript
\usepackage[utf8]{inputenc}
\usepackage{color}
\usepackage[unicode=true]{hyperref}
\hypersetup{breaklinks=true,
            bookmarks=true,
            pdfauthor={Patrik Jansson et al.},
            pdftitle={GRACeFUL D4.2: A Domain Specific Language (DSL) for GRACeFUL Concept Maps},
            colorlinks=true,
            citecolor=blue,
            urlcolor=blue,
            linkcolor=magenta,
            pdfborder={0 0 0}}
\urlstyle{same}  % don't use monospace font for urls
\setlength{\parindent}{0pt}
\setlength{\parskip}{4pt plus 2pt minus 1pt}
\setlength{\emergencystretch}{3em}  % prevent overfull lines

%% Cezar
\usepackage[margin=1.60in]{geometry}
\usepackage[verbose]{wrapfig}
\usepackage{graphicx}
\usepackage{subfig}
\usepackage{rotating}
\usepackage{lscape}
\usepackage{float}
\usepackage{geometry}
\usepackage{framed}
\usepackage{xspace}
\usepackage{acronym}

\definecolor{GRACeFULblue}{rgb}{0.20,0.60,0.86}

\newcommand{\grace}{GRACeFUL\xspace}
\acrodef{GCM}{\grace Concept Map}
\acrodef{GMB}{Group Model Building}
\acrodef{DSL}{Domain Specific Language}
\acrodef{CFP}{Constraint Functional Program}
\acrodef{RAT}{Rapid Assessment Tool}
\acrodef{CRUD}{Climate Resilient Urban Design}
\acrodef{CLD}{Causal Loop Diagram}
\hyphenation{GRACeFUL}

\author{}
\date{}

\begin{document}

\begin{center}
\includegraphics[width=5cm]{../coverpage/GRACeFULlogo.png}

\textcolor{GRACeFULblue}{Global systems Rapid Assessment tools\\
through Constraint FUnctional Languages}

\vspace{1cm}

FETPROACT-1-2014 Grant Nº 640954

\end{center}

\begin{framed}
\begin{center}
\Large
A Domain Specific Language (DSL) for \\
GRACeFUL Concept Maps\\[1ex]

D4.2\\[1ex]

\end{center}
\end{framed}

\vspace{1cm}

\noindent
\begin{tabular}{@{}ll@{}}
  Lead Participant:       & Chalmers (M. Algehed, S. Einarsdóttir, A. Gerdes, and P. Jansson)
\\Partners Contributing:  & KU Leuven, UPC
\\Dissemination Level:    & PU
\\Document Version:       & Draft (version 2017-01-19)
\\Date of Submission:     & 2017-01-31
\\Due Date of Delivery:   & 2017-01-31
\end{tabular}

\newpage

\section*{A Domain Specific Language (DSL) for\\GRACeFUL Concept Maps}\label{DSL4GRACeFUL}

\vfill

\tableofcontents

\vfill

\newpage

\section*{Abstract}\label{abstract}

This second deliverable of WP4 includes a short overview of a Domain
Specific Language (DSL) for description of GRACeFUL concept maps, and
a formal semantics in terms of types and functions. This is a
continuation of the initial work described in ``D4.1 Formal Concept
Maps Elements Descriptions'' delivered in project month 6. 


\subsection{GRACeFUL software architecture}

The software stack of the GRACeFUL project consists of
%
a visual editor frontend,
%
a network layer,
%
a GCM component library,
%
a DSL called GRACe,
%
a middleware called haskelzinc, and
%
a choice of external constraint solver.
%
In this section we briefly describe the different layers and how testing and
verification is performed on each layer.

\subsubsection*{Visual Editor}

The top layer of the software stack is the visual editor.
%
It provides a graphical user interface where the user can build GCMs
as a graphical map from available components.
%
The visual editor is implemented in the programming language JavaScript, using
the Data Driven Documents library (D3.js).
%

The code for the visual editor can be found in the
\href{https://github.com/GRACeFUL-project}{GRACeFUL-project} GitHub
repository
\href{https://github.com/GRACeFUL-project/GRACeFULEditor}{GRACeFULEditor}.

\todo{Is there anything to say about testing here? Maybe refer to user testing
  related deliverable/stuff}
\todo{AG: mention the evaluation with stakeholders here?}

The visual editor is described in deliverabe D3.3 including how it can be
accessed\footnote{\url{http://vocol.iais.fraunhofer.de/graceful-rat/static/}},
how it can be used, and more details on how it is implemented.

\todo{Refer to section on using type systems for testing/verification (and mention our
typed libraries as an example in that section.)}
\subsubsection*{GRACe}

GRACe is a domain specific language embedded in Haskell. It is used to express
GRACeFUL concept maps (GCMs) and GCM library components. GRACe programs
representing GCMs are compiled to haskelzinc constraint programs, and the
resulting solutions are passed back to GRACe. The DSL is described in deliverable
D4.2.

\subsubsection*{GCM component libraries}

The visual editor allows users to create a GRACeFUL Concept Map with components
of a chosen library. Components are written in GRACe and are grouped in a
library, for example a library with CRUD components. The visual editor can query
the components available in a library using its identifier. A JSON
representation of the components are sent to the visual editor.

The following code shows an example CRUD library, where |rain| and |pump| are
|GCM| components:
\begin{haskellcode}
library :: Library
library = Library "crud"
    [ item "rain" $
        rain ::: "amount" #
          tFloat .-> tGCM          (tPort $ "rainfall" # tFloat)
    , item "pump" $
        pump ::: "capacity" #
          tFloat .-> tGCM (tPair   (tPort $ "inflow"   # tFloat)
                                   (tPort $ "outflow"  # tFloat))
    ]
\end{haskellcode}
The example library makes use of Typed Values, which are explained in
Section~\ref{sec:verification}.

\todo{Say something about session check stuff.}
\subsubsection*{Communication with visual editor}

The GRACeServer provides a RESTful API and uses JSON\footnote{JavaScript Object
Notation} as the exchange format. The server offers the following webservices:
\begin{quote}
\begin{description}
\item [\haskell{libraries}] returns a list with the available component libraries
\item [\haskell{library (id)}] has a library identifier parameter and returns 
  a list with a description (in JSON) of the all library components
\item [\haskell{submit (gcm)}] takes a GRACeFUL Concept Map description as
  argument, which is translated to a GRACe DSL program, and returns the result 
  of the constraint solver
\end{description}
\end{quote}
The visual editor, developed in work package 3, communicates with the GRACeServer
by making service calls. The GRACeServer is available as the \texttt{RestAPI} 
executable in the \href{https://github.com/GRACeFUL-project/GRACe}{GRACe} repository.

\todo{Refer to GCMP section}
\subsubsection*{Haskelzinc}

Haskelzinc is a Haskell interface to the MiniZinc constraint
programming language.
%
It provides a Haskell DSL for building MiniZinc model representations and a parser that returns a representation of the solutions obtained
by running the MiniZinc model.

The latest version of haskelzinc (currently 0.3) is available from
\url{https://hackage.haskell.org/package/haskelzinc}.

\todo{Refer to section on inductive testing/backend comparison/ satisfiability testing}

% Front-end URL: http://vocol.iais.fraunhofer.de/graceful-rat/static/index.html
% TODO (after 2017-11-17 when the front end functionality should be more stable): take screen shot(s) showing some model


\subsection{Scope and limitations}

This deliverable (D4.4) covers the software technology side of testing
and verification of RATs.
%
The CRUD RAT evaluation based on stakeholder sessions is reported
elsewhere [D2.6 Evaluation Report CRUD RATs: m36] as is the
interactivity and usability of the visual front-end [T3.4, D3.3 VA EDA
  Tool Prototype (RAT components): m34].

\subsection{Rapid Assessment Tools}

Rapid assessment tools are used in large organisations like the World
Bank and the United Nations to assess risks and needs and to make
plans for improvement.
% http://www.wahooas.org/mshdvd2/assess_tools_MWL_Eng.htm
% http://www.fao.org/docrep/015/i2495e/i2495e06.pdf
In GRACeFUL the main focus is on Climate Resiliant Urban Design, but
the software developed in the project could be used for almost any
Global Systems Science problem.


\todo{Patrik+Solrun: describe our prototype RAT: screen shot}


\section{Background}
\label{sec:background}

We provide some background information about several aspects we use in
later sections, such that they become more accessible.
%
Whenever possible we give links for further reading as well.

\paragraph{\acl{DSL}} \citet{fowler} defines a \ac{DSL} as follows:
%
\emph{a computer programming language of limited expressiveness
  focussed on a particular domain}.
%
A \ac{DSL} is targetted at a specific class of programming tasks.
%
By restricting scope to a particular domain, one can tailor the
language specifically for that domain.
%
There are two main approaches to implementing \acp{DSL}:
\begin{description}
\item[standalone] A language with their own specialised syntax and
  parser to translate programs written in the \ac{DSL} into the host
  language.
%
  An advantage is that the syntax can be tailor made for the target
  audience, and does not have to resemble the host language.
%
  However, creating such a \ac{DSL} is labour intensive and it cannot
  easily reuse features, such as variables and conditionals, from the
  host language.
\item[embedded] An embedded language tries to offer a convenient
  syntax and abstraction mechanisms, but is offered as a library
  written in the host language.
%
  This means that all the existing facilities, such as abstractions
  standard libraries, are directly available.
%
  A disadvantage of an embedded \ac{DSL} is that the users may be
  unfamiliar with the host language.
\end{description}
\citet{Gibbons2015} gives a good overview of implementing \acp{DSL}
using functional programming.

% \paragraph{Semi-qualitative modelling} \todo{don't know if we should explain
% this here}

\paragraph{Constraint programming} Similar to functional and logic
programming, constraint programming is a declarative programming
paradigm, which means that the order of processing is not fixed.
%
A constraint program is formulated in terms of a number of
constraints.
%
Such constraints are different from the common primitives of
imperative programming languages in that they do not specify a step or
sequence of steps to execute, but rather the properties of a solution
to be found.
%
The program that identifies the solutions satisfying these constraints
is called a onstraint solver.
%
In general there might be none, one, or many solutions to a particular
contraint problem.

The constraint programming approach is to search for a state in which
a large number of constraints are satisfied at the same time.
%
A problem is typically expressed as a state containing a number of
unknown variables.
%
The constraint solver searches for values for all the variables.
%
For more information, see, e.g., \url{http://constraint.org}.

\paragraph{\acfp{GCM}}

%Factors, Criteria, Actions

%Outcomes: help the stakeholders

%\begin{itemize}
%\item find alternative policies that can satisfy the goals,
%\item identify inconsistencies in the problem definition,
%\item  present visualisations of possible future trajectories.
%\end{itemize}

In a \acl{GMB} session stakeholders perform a policy analysis that
results in the form of a \acl{GCM}.
%
This concept map contains important elements of a policy problem
definition: goals, criteria for assessing the achievement of those
goals, a description of the system involving factors and criteria, and
the competing alternatives.

The term \emph{factors} refers to characteristics of a system that can
take a value, either quantitative or qualitative, that can change over
time (source: the \emph{GRACeFUL Glossary of Terms}).
%
Similarly, \emph{external factors} correspond to \emph{inputs} to the
system.
%
As such, factors are assumed to be associated with measurable
values.
%
As in systems theory, a distinction is made between inputs that are
beyond the control of the system, and which are potentially uncertain
(or even unknown), and \emph{actions} or combinations of actions,
which represent the system theoretical \emph{controls}.

The interaction between the factors, and between the factors and the
criteria, represents a first description of the system.
%
This description is in the form of a stock-and-flow diagram (which is
a generalisation of \aclp{CLD}~\cite{burns}).
%
A causal loop diagram is a form of \emph{directed, simple, labelled
  graph}, i.e.:

TODO: AG: I have come this far.

\begin{enumerate}
\def\labelenumi{\alph{enumi}.}
\item
  a link between two nodes has a well-defined source and a target (it is
  \emph{directed}); intuitively, the link represents a causal relation
  between the source (cause) and the target (effect);
\item
  there can be at most one connection from one node to another (the
  graph is \emph{simple}): either a causal relation exists, or it does
  not. Note, however, that since the graph is directed, there may be an
  inverse connection from the ``effect'' to the ``cause'', as in
  feedback loops;
\item
  the connections are either \emph{positive} (an increase in the value
  of the source causes an increase in the value of the target), or
  \emph{negative} (an increase causes a decrease), hence the graph is
  \emph{labelled}.
\end{enumerate}

It can be seen from the last point that nodes \emph{must} represent
values that can be partially ordered. This rules out nodes that
represent, say, stakeholders.

The GRACeFUL system can assist in building the causal loop diagram, for
example by suggesting connections, or by pointing out inconsistencies.

At this stage, if we can attribute qualitative values to the various
nodes, the causal loop diagram describes a system that can make the
object of a qualitative simulation. The criteria play the roles of
constraints: the qualitative trajectories that lead outside the criteria
can be pruned by the constraint programming system.

The causal loop diagram is enhanced to a GRACeFUL concept map by the
addition of extra nodes and non-causal links. The extra nodes represent
actions, goals, alternatives, and stakeholders.

Like all previous elements, the actions are also obtained from the
stakeholders. As explained in the previous section, actions are the
components of alternatives, among which will be found the policy
solutions we are seeking. Alternatives are, according to the GRACeFUL
Glossary of Terms, ``action or combination of actions that influence the
values of criteria''. They correspond to Walker's ``alternative
policies'', where a \emph{policy} is defined ``loosely'' as ``a set of
actions taken to solve a problem''. As in the case of goal definition,
the GRACeFUL system can suggest such ``atomic actions'' that are useful
to the goals in the context of the problem at hand.

The role of actions in the GRACeFUL concept maps is to give initial
values to the factors and criteria they influence. They do not link via
causal loops to these factors, since actions are not of the requisite
type (they do not represent values that can increase or decrease).
However, they will usually have non-causal links, representing
constraining relationships to the factors.

In the other cases of extra nodes (goals, alternatives, and
stakeholders), the links represent primarily ``belongs to''
relationships: there are links from stakeholders to goals, indicating
which stakeholders have declared which goals; there are links from goals
to the criteria by which they are assessed; finally, there are links
between alternatives and the factors and criteria that they (i.e., their
constituent atomic actions) influence.


\section{GenericLibrary: a DSL for GCMs}
\label{sec:gl}


TODO: quote short parts of D4.1 and highlight similarities and differences
\todo{Make nicer}
GenericLibrary, henceforth abbreviated to GL, is a DSL for descriptions
of GRACeFUL concept map components in Haskell. The DSL addresses the issue
of bridging the gap between constraint programming and the visualisation
layer by providing abstractions for modular constraint programming.

The DSL is divided in to two parts. The first part, \texttt{GCM}, allows the user to describe the
interactions of GRACeFUL concept map components and has facilities for
constructing new components from existing ones. The second part,
\texttt{CP}, features primitives for constructing constraint programs
which describe the behaviour of an individual components.

\todo{This section needs clarification}
The architecture is demonstrated by the example below. A \texttt{GCM}
component exposing a single port is defined, it contains a small \texttt{CP} program (in the
\texttt{component} block) which describes the local behaviour of the
component and features a local variable which is unreachable from the
outer, structural, description.

\todo{if possible, make this example nicer}
\begin{verbatim}
gcmComponent :: GCM (Port Float)
gcmComponent = do
  externalPort <- createPort
  component $ do
   variable <- createLVar
   assert (externalPort === variable + 1)
  return externalPort
\end{verbatim}

\todo{How do we describe the available language constructs?}

\subsection{The language}\todo{Something about Haskell metaprogramming}
\todo{flesh out?}
As noted earlier, the GL language is primarily constructed from two different languages,
GCM and CP.

\todo{Something about how to include a CP component in a GCM one?}
The GCM language supports reasoning about components and the interfaces between components,
with language constructs such as \texttt{createPort} for creating interfaces and \texttt{link}
as well as \texttt{linkBy} for connecting interfaces to each other.

The CP language supports reasoning about integer and floating-point arithmetic, boolean expressions,
and arrays. It has constructions like \texttt{createLVar} and \texttt{assert} for reasoning about
constraint variables.

As can be seen there is a clear separation of concerns in GL between the high-level primitives
for constructing complicated components from simpler ones, represented in GCM, and the low level
implementation details with which the CP language is concerned.


% Let us start with concepts and factors, which are, in a sense, the
% simplest elements of the GRACeFUL concept maps. Concepts are high-level
% descriptions of the components of a system, and they are refined into
% factors, which, as we have seen, determine the state of the system. The
% most important feature of a factor is that it is \emph{measurable}: it
% can take one value from a set of values. These values must be ordered,
% otherwise we do not have a notion of increasing or decreasing, as
% required by the use of factors in causal loop diagrams.
%
% Different factors can be associated with the same type of value. For
% example, there might be factors that represent temperatures in different
% parts of the system, such as ``water temperature'' and ``air
% temperature''. Even if the values of water temperature and air
% temperature happen to be equal, the two factors are still different.
% That means that a defining component of a factor is its identifier, its
% name. This suggests formalising the notion of a factor as a pair
% consisting of a name and a value. Such a formalisation makes explicit
% the part that stays the same, the name, in addition to the part that
% changes, the value. This is a common way of modelling states in
% functional programming.
%
% On the other hand, we can certainly create a causal loop diagram
% without knowing the value associated to the factors. None of the
% factors in the causal loops in Figures \ref{fig:cl1} and \ref{fig:cl2}
% have values associated with them. Therefore, we must formalise factors
% as pairs consisting of a name and \emph{potentially} of a value of an
% ordered type:
%
% \begin{Shaded}
% \begin{Highlighting}[]
% \KeywordTok{data} \NormalTok{(}\DataTypeTok{Ord} \NormalTok{value) }\OtherTok{=>} \DataTypeTok{Factor} \NormalTok{value  }\FunctionTok{=}  \DataTypeTok{MkFactor} \DataTypeTok{Name} \NormalTok{(}\DataTypeTok{Maybe} \NormalTok{value)}
% \end{Highlighting}
% \end{Shaded}
%
% For the moment, we can take the names of factors to be just strings:
%
% \begin{Shaded}
% \begin{Highlighting}[]
% \KeywordTok{type} \DataTypeTok{Name}  \FunctionTok{=}  \DataTypeTok{String}
% \end{Highlighting}
% \end{Shaded}
%
% Since we aim for a system based on qualitative reasoning, most of the
% types of value we shall use are going to be finite types, consisting of
% an enumeration of the symbols for possible landmark values and for the
% intervals delimited by the landmark values. For example, a very common
% type of qualitative values is:
%
% \begin{Shaded}
% \begin{Highlighting}[]
% \KeywordTok{data} \DataTypeTok{QualitativeValue}        \FunctionTok{=}  \DataTypeTok{Negative} \FunctionTok{|} \DataTypeTok{Zero} \FunctionTok{|} \DataTypeTok{Positive}
%                                 \KeywordTok{deriving} \NormalTok{(}\DataTypeTok{Eq}\NormalTok{, }\DataTypeTok{Ord}\NormalTok{)}
% \end{Highlighting}
% \end{Shaded}
%
% Here we have one landmark value, \VERB|\DecValTok{0}|, denoted by
% \VERB|\DataTypeTok{Zero}|, and two intervals:
% \VERB|\NormalTok{(}\FunctionTok{-}\NormalTok{inf, }\DecValTok{0}\NormalTok{)}|,
% denoted \VERB|\DataTypeTok{Negative}|, and
% \VERB|\NormalTok{(}\DecValTok{0}\NormalTok{, inf)}|, denoted
% \VERB|\DataTypeTok{Positive}|.
%
% Continuing the example of temperature factors above, we can have:
%
% \begin{Shaded}
% \begin{Highlighting}[]
% \NormalTok{t1, t2}\OtherTok{                 ::}  \DataTypeTok{Factor} \DataTypeTok{QualitativeValue}
% \NormalTok{t1                      }\FunctionTok{=}  \DataTypeTok{MkFactor} \StringTok{"Water Temperature"} \DataTypeTok{Nothing}
% \NormalTok{t2                      }\FunctionTok{=}  \DataTypeTok{MkFactor} \StringTok{"Air Temperature"} \NormalTok{(}\DataTypeTok{Just} \DataTypeTok{Positive}\NormalTok{)}
% \end{Highlighting}
% \end{Shaded}
%
% Here, \VERB|\NormalTok{t1}| represents the factor ``Water Temperature'',
% which does not currently have a defined value, and \VERB|\NormalTok{t2}|
% represents the factor ``Air Temperature'', whose current value is
% \VERB|\DataTypeTok{Positive}|.
%
% Factors are introduced via the refinement of \emph{concepts}. However,
% once the factors have been defined, concepts play no further role: in
% fact, they do not even appear in the current versions of the GRACeFUL
% concept maps. This is not surprising: concepts are high-level
% descriptions of parts of systems. In mathematical systems theory, (parts
% of) systems are \emph{defined} by their states, and so concepts are
% defined by the factors that they are ``refined'' into in the course of
% stakeholder interactions. Moreover, it is not clear that it is useful to
% distinguish between two concepts that are refined in identical sets of
% factors. Still, at this stage we think it useful to keep a record of the
% concept name, for example for the purpose of looking it up in a database
% of pre-defined (or pre-refined) concepts (thus creating re-usable
% building blocks). Formally, we have:
%
% \begin{Shaded}
% \begin{Highlighting}[]
% \KeywordTok{data} \DataTypeTok{Concept} \NormalTok{value        }\FunctionTok{=}  \DataTypeTok{MkConcept} \DataTypeTok{Name} \NormalTok{[}\DataTypeTok{Factor} \NormalTok{value]}
% \end{Highlighting}
% \end{Shaded}
%
% For simplicity, we have assumed that all the factors defining a concept
% have the same sets of possible values, in order to overcome some
% limitations of our Haskell notation.
%
% We now move on to the formalisation of goals and criteria. Criteria are
% similar to factors: they are part of the causal loop diagrams, hence
% have values in an ordered set, names, and maybe a definite value.
% Additionally, however, criteria are associated to \emph{predicates},
% which tell us whether a given value satisfies a concrete criterion or
% not:
%
% \begin{Shaded}
% \begin{Highlighting}[]
% \KeywordTok{type} \DataTypeTok{Predicate} \NormalTok{a    }\FunctionTok{=}  \NormalTok{(a }\OtherTok{->} \DataTypeTok{Bool}\NormalTok{)}
% \end{Highlighting}
% \end{Shaded}
%
% Therefore, we can formalise criteria as
%
% \begin{Shaded}
% \begin{Highlighting}[]
% \KeywordTok{data}  \NormalTok{(}\DataTypeTok{Ord} \NormalTok{value) }\OtherTok{=>} \DataTypeTok{Criterion} \NormalTok{value  }\FunctionTok{=}
%       \DataTypeTok{MkCriterion} \DataTypeTok{Name} \NormalTok{(}\DataTypeTok{Maybe} \NormalTok{value) (}\DataTypeTok{Predicate} \NormalTok{value)}
% \end{Highlighting}
% \end{Shaded}
%
% (Alternatively, we could have extended the type
% \VERB|\DataTypeTok{Factor} \NormalTok{value}|.)
%
% For example:
%
% \begin{Shaded}
% \begin{Highlighting}[]
% profit                  \OtherTok{::}  \DataTypeTok{Criterion} \DataTypeTok{QualitativeValue}
% \NormalTok{profit                   }\FunctionTok{=}  \DataTypeTok{MkCriterion} \StringTok{"Profit"} \NormalTok{(}\DataTypeTok{Just} \DataTypeTok{Zero}\NormalTok{) isPositive}
%                             \KeywordTok{where}
%                             \NormalTok{isPositive x }\FunctionTok{=} \NormalTok{x }\FunctionTok{>} \DataTypeTok{Zero}
% \end{Highlighting}
% \end{Shaded}
%
% introduces a criterion formalising ``profits should be (strictly)
% positive'', where the current value, \VERB|\DataTypeTok{Zero}|, does not
% satisfy the criterion.
%
% Criteria result from the refinement of goals, just as factors result
% from the refinement of concepts. It is no surprise, therefore, that the
% formalisation of goals will be just like that of concepts:
%
% \begin{Shaded}
% \begin{Highlighting}[]
% \KeywordTok{data} \DataTypeTok{Goal} \NormalTok{value        }\FunctionTok{=}  \DataTypeTok{MkGoal} \DataTypeTok{Name} \NormalTok{[}\DataTypeTok{Criterion} \NormalTok{value]}
% \end{Highlighting}
% \end{Shaded}
%
% At this point, it is useful to briefly discuss an essential difference
% between \emph{formalisation} and \emph{implementation}. The
% implementation of goals in the GRACeFUL system will very likely be much
% more complex than the formalisation presented here. We mentioned the
% possibility of looking up goals in a library of pre-defined goals, in
% order to assist the stakeholders in their search for adequate criteria.
% We might have various kinds of pre-defined goals, which might themselves
% be related in complicated ways, for example in order to assess
% inconsistencies between them. We could use \emph{ontologies} for the
% representation of goals, (and, for that matter, also concepts, factors,
% criteria, etc.), linking to Haskell RDF libraries, etc. Therefore, the
% implementation of \VERB|\DataTypeTok{Goal}| will likely result in a more
% complex datatype, such as:
%
% \begin{Shaded}
% \begin{Highlighting}[]
% \KeywordTok{data} \DataTypeTok{Goal}            \FunctionTok{=}  \DataTypeTok{GoalRDF} \DataTypeTok{Triple}
%                      \FunctionTok{|}  \DataTypeTok{G1} \DataTypeTok{PreDefinedG1}
%                      \FunctionTok{|}  \DataTypeTok{G2} \DataTypeTok{PreDefinedG2}
%                      \FunctionTok{|}  \FunctionTok{...}
%                      \FunctionTok{|}  \DataTypeTok{FreeFormGoal} \DataTypeTok{String}
% \end{Highlighting}
% \end{Shaded}
%
% Here, we are only concerned with \emph{understanding} as clearly as
% possible the notion of goal, factor, etc. as pertaining to their use in
% the GRACeFUL concept maps. We do not search for the most efficient
% representation in computational terms. We are also stripping away
% anything that does not belong to the essence of these notions \emph{as
% used in the concept maps}. One reason for this is that the formalisation
% undertaken here will be part of the \emph{specifications} to the
% implementations carried out in the course of the project. The simpler
% the specification, the likelier it is that it will be correctly
% understood and implemented. Moreover, since the specification is that
% against which we measure the correctness of the implementation, it is
% important that we do not overly constrain this implementation, for
% example by deciding that goals are going to be represented via RDF
% triples. At this stage, such a decision would be largely arbitrary.
%
% Now that we have formalised factors and criteria, the nodes of the
% causal loop diagrams, we can formalise these diagrams themselves. As
% mentioned above, these are instances of simple, directed graphs with
% labelled connections. The following datatype represents a simple
% directed graph, parametrised on the types of nodes and connections:
%
% \begin{Shaded}
% \begin{Highlighting}[]
% \KeywordTok{data} \DataTypeTok{Graph} \NormalTok{node conn  }\FunctionTok{=}  \DataTypeTok{MkGraph} \NormalTok{((node, node) }\OtherTok{->} \DataTypeTok{Maybe} \NormalTok{conn)}
% \end{Highlighting}
% \end{Shaded}
%
% In (other) words, a graph is given by a function which, to every ordered
% pair of nodes (a source and a target), associates either no connection,
% or just one connection.
%
% The connections in a causal loop diagram are labelled either with a
% plus, or a minus:
%
% \begin{Shaded}
% \begin{Highlighting}[]
% \KeywordTok{data} \DataTypeTok{Connection}       \FunctionTok{=}  \DataTypeTok{Plus} \FunctionTok{|} \DataTypeTok{Minus}
% \end{Highlighting}
% \end{Shaded}
%
% The nodes of our causal loop diagrams are either factors or criteria:
%
% \begin{Shaded}
% \begin{Highlighting}[]
% \KeywordTok{type} \DataTypeTok{Node} \NormalTok{value       }\FunctionTok{=}  \DataTypeTok{Either} \NormalTok{(}\DataTypeTok{Criterion} \NormalTok{value) (}\DataTypeTok{Factor} \NormalTok{value)}
% \end{Highlighting}
% \end{Shaded}
%
% Therefore, we can formalise a causal loop as
%
% \begin{Shaded}
% \begin{Highlighting}[]
% \KeywordTok{type} \DataTypeTok{CausalLoop} \NormalTok{value  }\FunctionTok{=}  \DataTypeTok{Graph} \NormalTok{(}\DataTypeTok{Node} \NormalTok{value) }\DataTypeTok{Connection}
% \end{Highlighting}
% \end{Shaded}
%
% GRACeFUL concept maps represent an enrichment of the causal loops with
% nodes representing the possible (atomic) actions and with nodes and
% links representing constraints between actions and factors.
%
% Actions can influence the nodes of a causal loop diagram, by changing
% the value associated with that node. We can formalise such an influence
% as a function that can return a new value for a given node (or
% \VERB|\DataTypeTok{Nothing}| in the case of ``no influence'').
%
% \begin{Shaded}
% \begin{Highlighting}[]
% \KeywordTok{type} \DataTypeTok{Influence} \NormalTok{value  }\FunctionTok{=}  \DataTypeTok{Node} \NormalTok{value }\OtherTok{->} \DataTypeTok{Maybe} \NormalTok{value}
% \end{Highlighting}
% \end{Shaded}
%
% An action is then defined as a named influence:
%
% \begin{Shaded}
% \begin{Highlighting}[]
% \KeywordTok{data} \DataTypeTok{Action} \NormalTok{value     }\FunctionTok{=}  \DataTypeTok{MkAction} \DataTypeTok{Name} \NormalTok{(}\DataTypeTok{Influence} \NormalTok{value)}
% \end{Highlighting}
% \end{Shaded}
%
% An alternative policy is, similar to a goal or a concept, a named set of
% actions:
%
% \begin{Shaded}
% \begin{Highlighting}[]
% \KeywordTok{data} \DataTypeTok{Alternative}  \NormalTok{value    }\FunctionTok{=}  \DataTypeTok{MkAlternative} \DataTypeTok{Name} \NormalTok{[}\DataTypeTok{Action} \NormalTok{value]}
% \end{Highlighting}
% \end{Shaded}
%
% One of the major outputs of the GRACeFUL system will be a list of
% alternatives which meet the goals and the constraints; we could define
%
% \begin{Shaded}
% \begin{Highlighting}[]
% \KeywordTok{type} \DataTypeTok{GRACeFUL_Solution} \NormalTok{value  }\FunctionTok{=}  \NormalTok{[}\DataTypeTok{Alternative} \NormalTok{value]}
% \end{Highlighting}
% \end{Shaded}
%
% As we have seen in the previous section, actions can be constrained by
% factors (e.g., investments are constrained by the available capital).
% This is common in systems theory, where the set of controls that can be
% used when the system is in a given state depend on that state.
%
% Similarly, there can be constraint relations between the factors
% themselves. These constraints can either be associated with concepts, as
% in the case in which the constraints relate factors stemming from the
% refinement of the same concept, or with interactions between concepts.
%
% The constraint relations are, in general, non-causal, and are thus
% genuine enhancements over a causal loop diagram. They can sometimes be
% represented by directed links, for example, ``source value may not
% exceed target value'', but in general they will require the introduction
% of a new kind of node. It is easy to see that a constraint such as
% ``only one of these three factors may be negative'' is not expressible
% as a link between two factors.
%
% Constraints that can be expressed as links can be formalised as
%
% \begin{Shaded}
% \begin{Highlighting}[]
% \KeywordTok{data} \DataTypeTok{ConstraintLink} \NormalTok{value  }\FunctionTok{=}  \DataTypeTok{MkConstraintLink} \NormalTok{(}\DataTypeTok{Predicate} \NormalTok{(value, value))}
% \end{Highlighting}
% \end{Shaded}
%
% This corresponds to the intuition that we can model a constraint as a
% link only if it is a binary relation.
%
% For example:
%
% \begin{Shaded}
% \begin{Highlighting}[]
% doesNotExceed             \OtherTok{::}  \DataTypeTok{Ord} \NormalTok{value }\OtherTok{=>} \DataTypeTok{ConstraintLink} \NormalTok{value}
% \NormalTok{doesNotExceed              }\FunctionTok{=}  \DataTypeTok{MkConstraintLink} \NormalTok{gt}
%                               \KeywordTok{where}
%                               \NormalTok{gt (v1, v2) }\FunctionTok{=} \NormalTok{v1 }\FunctionTok{<=} \NormalTok{v2}
% \end{Highlighting}
% \end{Shaded}
%
% Constraints between more than two factors cannot be modelled as links.
% For these, we need to introduce nodes which collect the values of
% several factors or actions, and check if these values satisfy the
% constraint, operating in a similar way to the criteria. As opposed to
% the criteria, the value associated to a constraint is either true, or
% false.
%
% In a binary relation such as \VERB|\NormalTok{doesNotExceed}|, we can
% distinguish between the two elements (``what does not exceed what?'')
% based on the source and target order of a directed link. This is not
% possible in the case of a constraint represented by a node. Therefore,
% the various roles of the related values must be indicated by the links.
% As an example, consider the constraint given by
%
% \begin{Shaded}
% \begin{Highlighting}[]
% inBetweenPred             \OtherTok{::}  \DataTypeTok{Ord} \NormalTok{value }\OtherTok{=>} \DataTypeTok{Predicate} \NormalTok{(value, value, value)}
% \NormalTok{inBetweenPred (v1, v2, v3) }\FunctionTok{=}  \NormalTok{v1 }\FunctionTok{<=} \NormalTok{v2 }\FunctionTok{&&} \NormalTok{v2 }\FunctionTok{<=} \NormalTok{v3}
% \end{Highlighting}
% \end{Shaded}
%
% This constraint will be represented by a node. There will be three links
% into this node, representing the three arguments. These links will need
% to be labelled, in order for the constraint node to be able to tell
% which one of the arguments is associated to each link. We can introduce
% a type for these labels:
%
% \begin{Shaded}
% \begin{Highlighting}[]
% \KeywordTok{data} \DataTypeTok{Role}                  \FunctionTok{=}  \DataTypeTok{V1} \FunctionTok{|} \DataTypeTok{V2} \FunctionTok{|} \DataTypeTok{V3}
%                               \KeywordTok{deriving} \DataTypeTok{Eq}
% \end{Highlighting}
% \end{Shaded}
%
% A role link will connect nodes to a constraint node, specifying their
% role:
%
% \begin{Shaded}
% \begin{Highlighting}[]
% \KeywordTok{data} \DataTypeTok{RoleLink} \NormalTok{role         }\FunctionTok{=}  \DataTypeTok{MkRoleLink} \NormalTok{role}
% \end{Highlighting}
% \end{Shaded}
%
% Finally, a constraint node will be represented by a predicate on a list
% of role-value pairs:
%
% \begin{Shaded}
% \begin{Highlighting}[]
% \KeywordTok{data} \DataTypeTok{ConstraintNode} \NormalTok{role val  }\FunctionTok{=}  \DataTypeTok{MkConstraintNode} \NormalTok{(}\DataTypeTok{Predicate} \NormalTok{[(role, val)])}
% \end{Highlighting}
% \end{Shaded}
%
% For example, we can define the node associated to the
% \VERB|\NormalTok{inBetweenPred}| above by
%
% \begin{Shaded}
% \begin{Highlighting}[]
% inBetweenC                \OtherTok{::}  \DataTypeTok{Ord} \NormalTok{value }\OtherTok{=>} \DataTypeTok{ConstraintNode} \DataTypeTok{Role} \NormalTok{value}
% \NormalTok{inBetweenC                 }\FunctionTok{=}  \DataTypeTok{MkConstraintNode} \NormalTok{p}
%                               \KeywordTok{where}
%                               \NormalTok{p as }\FunctionTok{=} \NormalTok{inBetweenPred (toTriple as)}
% \end{Highlighting}
% \end{Shaded}
%
% (we omit the implementation of \VERB|\NormalTok{toTriple}|).
%
% There is one remaining element of the GRACeFUL concept map as presented
% above, namely the nodes representing the stakeholders. These are meant
% to relate the stakeholders to their goals, thus we can formalise them
% as:
%
% \begin{Shaded}
% \begin{Highlighting}[]
% \KeywordTok{data} \DataTypeTok{Stakeholder} \NormalTok{value  }\FunctionTok{=}  \DataTypeTok{MkStakeholder} \DataTypeTok{Name} \NormalTok{[}\DataTypeTok{Goal} \NormalTok{value]}
% \end{Highlighting}
% \end{Shaded}
%
% The nodes of the GRACeFUL concept maps are therefore:
%
% \begin{Shaded}
% \begin{Highlighting}[]
% \KeywordTok{data} \DataTypeTok{ConceptMapNode} \NormalTok{role value  }\FunctionTok{=}  \DataTypeTok{CN} \NormalTok{(}\DataTypeTok{Node} \NormalTok{value)}
%                                 \FunctionTok{|}  \DataTypeTok{AN} \NormalTok{(}\DataTypeTok{Action} \NormalTok{value)}
%                                 \FunctionTok{|}  \DataTypeTok{GN} \NormalTok{(}\DataTypeTok{Goal} \NormalTok{value)}
%                                 \FunctionTok{|}  \DataTypeTok{CO} \NormalTok{(}\DataTypeTok{ConstraintNode} \NormalTok{role value)}
%                                 \FunctionTok{|}  \DataTypeTok{SN} \NormalTok{(}\DataTypeTok{Stakeholder} \NormalTok{value)}
% \end{Highlighting}
% \end{Shaded}
%
% Besides the causal connections in the causal loop diagram, we also have
% links from actions to the factors and criteria they influence, from the
% criteria to the goals they assess, from stakeholders to the goals they
% ``own'', constraint links, and role links.
%
% \begin{Shaded}
% \begin{Highlighting}[]
% \KeywordTok{data} \DataTypeTok{ConceptMapConnection} \NormalTok{role value  }\FunctionTok{=}  \DataTypeTok{CC} \DataTypeTok{Connection}
%                                       \FunctionTok{|}  \DataTypeTok{ActionToNode}
%                                       \FunctionTok{|}  \DataTypeTok{CriterionToGoal}
%                                       \FunctionTok{|}  \DataTypeTok{CL} \NormalTok{(}\DataTypeTok{ConstraintLink} \NormalTok{value)}
%                                       \FunctionTok{|}  \DataTypeTok{RL} \NormalTok{(}\DataTypeTok{RoleLink} \NormalTok{role)}
%                                       \FunctionTok{|}  \DataTypeTok{StakeholderToGoal}
% \end{Highlighting}
% \end{Shaded}
%
% We can now define the GRACeFUL concept map as a simple, directed graph:
%
% \begin{Shaded}
% \begin{Highlighting}[]
% \KeywordTok{type} \DataTypeTok{ConceptMap} \NormalTok{role value  }\FunctionTok{=}  \DataTypeTok{Graph} \NormalTok{(}\DataTypeTok{ConceptMapNode} \NormalTok{role value)}
%                                      \NormalTok{(}\DataTypeTok{ConceptMapConnection} \NormalTok{role value)}
% \end{Highlighting}
% \end{Shaded}
%
% A concept map
% \VERB|\NormalTok{cm }\FunctionTok{=} \DataTypeTok{MkGraph} \NormalTok{f}|
% is \emph{valid} if:
%
% \begin{itemize}
% \item
%   \VERB|\NormalTok{f (cn1, cn2) }\FunctionTok{=} \DataTypeTok{ActionToNode}|
%   iff \VERB|\NormalTok{cn1}| is of the form
%   \VERB|\DataTypeTok{AN} \NormalTok{(}\DataTypeTok{MkAction} \NormalTok{name infl)}|,
%   \VERB|\NormalTok{cn2}| is of the form
%   \VERB|\DataTypeTok{CN} \NormalTok{node}|, and
%   \VERB|\NormalTok{infl node}| is not \VERB|\DataTypeTok{Nothing}|.
% \item
%   \VERB|\NormalTok{f (cn1, cn2) }\FunctionTok{=} \DataTypeTok{CriterionToGoal}|
%   iff \VERB|\NormalTok{cn1}| is of the form
%   \VERB|\DataTypeTok{CN} \NormalTok{(}\DataTypeTok{Left} \NormalTok{crit)}|,
%   \VERB|\NormalTok{c2}| is of the form
%   \VERB|\DataTypeTok{GN} \NormalTok{(}\DataTypeTok{MkGoal} \NormalTok{name crits)}|,
%   and \VERB|\NormalTok{crit}| is an element of \VERB|\NormalTok{crits}|.
% \item
%   \VERB|\NormalTok{f (cn1, cn2) }\FunctionTok{=} \DataTypeTok{StakeholderToGoal}|
%   iff \VERB|\NormalTok{cn1}| is of the form
%   \VERB|\DataTypeTok{SN} \NormalTok{(}\DataTypeTok{MkStakeholder} \NormalTok{sname goals)}|,
%   \VERB|\NormalTok{cn2}| is of the form
%   \VERB|\DataTypeTok{GN} \NormalTok{goal}|, and \VERB|\NormalTok{goal}|
%   is an element of \VERB|\NormalTok{goals}|.
% \item
%   \VERB|\NormalTok{f (cn1, cn2) }\FunctionTok{=} \DataTypeTok{CC} \NormalTok{plusOrMinus}|
%   only if \VERB|\NormalTok{cn1}| is of the form
%   \VERB|\DataTypeTok{CN} \NormalTok{node1}| and \VERB|\NormalTok{cn2}|
%   is of the form \VERB|\DataTypeTok{CN} \NormalTok{node2}|.
% \item
%   \VERB|\NormalTok{f (cn1, cn2) }\FunctionTok{=} \DataTypeTok{RL} \NormalTok{r}|
%   only if \VERB|\NormalTok{cn1}| is a factor node or an action node, and
%   \VERB|\NormalTok{cn2}| is a constraint node.
% \item
%   \VERB|\NormalTok{f (cn1, cn2) }\FunctionTok{=} \DataTypeTok{CL} \NormalTok{cl}|
%   only if \VERB|\NormalTok{cn1}| and \VERB|\NormalTok{cn2}| are factors
%   or actions.
% \end{itemize}

\subsection{Example: Runoff flow}\label{example-runoff-flow}

\todo{Insert figure showing a graphical view?}

\subsection{Example: Runoff flow: DSL textual
input}\label{example-runoff-flow-dsl-textual-input}

\todo{Say something about more the collection of examples available online: }
The collection of examples is available in the \verb+examples/+ directory of the GitHub repository \href{https://github.com/GRACeFUL-project/GenericLibrary}{GenericLibrary}:

% -- A GCM representing a simluation of the swedish energy system
% energySystem :: GCM ()
% SmallExample.hs

\subsection{Example: Runoff flow}
\label{example-runoff-flow}

We show a small GL program which models a rain runoff area, like a
town square, which has been provided with a pump to alleviate possible
flooding issues (this is a common procedure in countries like the
Netherlands).
%
This example is a small part of a larger model used in the CRUD case
study meant to show how GL can be employed to model concrete problems
in CRUD.


\subsubsection{DSL textual input}
\label{example-runoff-flow-dsl-textual-input}

\todo{some explaining text?}

\begin{verbatim}
pump :: Float -> GCM (Port Float, Port Float)
pump ... -- As before

rain :: Float -> GCM (Port Float)
rain ... -- As before

storage :: Float -> GCM (Port Float, Port Float, Port Float)
storage cap = do
  inflow   <- createPort
  outlet   <- createPort
  overflow <- createPort
  component $ do
    currentStored <- createVariable
    inf <- value inflow
    out <- value outlet
    ovf <- value overflow
    sto <- value currentStored
    assert $ sto === inf - out - ovf
    assert $ sto `inRange` (0, lit cap)
    assert $ (ovf .> 0) ==> (sto === lit cap)
    assert $ ovf .>= 0
  return (inflow, outlet, overflow)

example :: GCM ()
example = do
  (inflowP, outflowP) <- pump 5
  (inflowS, outletS, overflowS) <- storage 4
  rainflow <- rain 10

  link inflowP outletS
  link inflowS rainflow

  output "Overflow" overflowS
\end{verbatim}

When we run the above program we get the output

\begin{verbatim}
ghci> runGCM example
{"Overflow" : 0}
\end{verbatim}
\todo{More explanation needed!}

\todo{Say something about actions!}

\subsection{Software infrastructure (around the
DSL)}\label{software-infrastructure-around-the-dsl}

\todo{Replace by a more current description of the software status}

\todo{Should we even describe the two different software stacks}

\begin{itemize}
\item
  (VIS layer)
\item
  GenericLibrary
\item
  (MiniZinc)
\end{itemize}
Another stack
\begin{itemize}
\item
  (VIS layer)
\item
  GraphDSL (describe the graph) + ConstraintDSL (still ongoing work)
\item
  QPNModeler: encodes the Qualitative Probabilistic Network semantics
\item
  haskelzinc: Haskell interface to (and from) MiniZinc
\item
  (MiniZinc)
\end{itemize}

\section{Installation and software requirements}
\label{install-and-reqs}

%% -- APPENDIX: Installation instructions -------------------------------------
%% ----------------------------------------------------------------------------

\section{Installation and usage}
\label{appendix-install}
%
In this section we outline the process of installing the GRACe software and its
required dependencies. We start with an overview of the software dependencies
required for developing GRACe programs (Appendix~\ref{install-overview}).
Following this, we provide installation instructions for a platform-independent
package built on the \href{https://www.docker.com/}{Docker} platform, intended
for users who only wish to execute a pre-existing example
(Appendix~\ref{install-docker}).

%% -- A: Software dependencies ------------------------------------------------

\subsection{Software dependencies}
\label{install-overview}
%
The GRACe language is an embedded domain-specific language implemented in the
\href{https://www.haskell.org/}{Haskell} programming language, and uses the
solver tools from the \href{http://www.minizinc.org/}{MiniZinc} software
distribution. Hence, development and execution of GRACe programs requires the
following software dependencies to be met:
%
\begin{itemize}
  \item[(i)] The MiniZinc and Gecode solver software.
  \item[(ii)] A Haskell toolchain able to download packages from
    \href{https://hackage.haskell.org/}{Hackage}, for instance
    \href{https://www.haskell.org/platform/}{the Haskell Platform}.
\end{itemize}

Detailed instructions for installing the Haskell Platform is available at
\url{https://www.haskell.org/platform/}. The preferred way of installing the
MiniZinc and Gecode components is by way of the bundled binary packages
available at \url{http://www.minizinc.org/software.html}.

Finally, instructions for setting up the GRACe library for building and
executing GRACe programs is available at the \href{https://github.com/GRACeFUL-%
project/GRACe/blob/master/Readme.md}{GRACe GitHub repository}.

%% -- A: Using Docker ---------------------------------------------------------

\subsection{Installation using Docker}
\label{install-docker}

In addition to the installation instructions for the GRACe development tools
(Appendix~\ref{install-overview}) we also provide a platform-independent Docker
image containing an executable for the \verb!OilCrops! example, written in
GRACe.

The \verb!OilCrops! example contains a small optimization problem in which the
objective is to dedicate a set amount of farmland area to three different crops,
with the goal of maximizing the yield of vegetable oil produced from these 
crops. A full description of the example can be found at \url{url.here}. 
\todo{Link to OilCrops.hs `explanation'.}
The GRACe source code of the example can be found at \url{https://github.com/%
GRACeFUL-project/GRACe/blob/master/examples/OilCrops.hs}.

Running the \verb!OilCrops! example requires the Docker Community Edition (CE)
to be installed. Docker CE, as well as installation instructions are available
at \url{https://www.docker.com/products/docker}. Once Docker CE is installed,
the example can be executed using the Docker application as follows. Open a
terminal (the command prompt for Windows users) and execute the commands
%
\begin{verbatim}
  docker pull eugraceful/grace-examples:latest
  docker run --rm eugraceful/grace-examples:latest
\end{verbatim}

\noindent
This will run the \verb!OilCrops! example and write the problem solution to
standard output.


\section{Formal semantics}
\label{sec:semantics}
% stock & flow - CLD simplification (specialization/simplification of GCM)
% Why formalize?
We would like to define formal semantics for our DSL in order to be
able to reason formally about it.
%
By formal reasoning we can confirm the DSL's robustness and gain
further insight into it.

We started by considering Causal Loop Diagrams, a specialization of
the GCMs our DSL describes.
%
This was done to simplify the initial scope of the work, with the
expectation that the semantics defined for CLDs could then be extended
to the more general GCMs.
%
Work to extend and generalize these semantics to describe
stock-and-flow diagrams, and thereby make them more consistent with
the current implementation of the DSL, is ongoing.

We have also modelled CLDs \emph{within} GRACe: the code for this
model can be found in the file \verb|QualitativeExample.hs|.

\subsection{Causal Loop Diagrams}
%
A Causal Loop Diagram (CLD) is a directed graph used to display causal
relationships between variables.
%
The vertices represent the variables and the edges represent
qualitative causal relationships, which can be positive or negative.

Different approaches can be taken in defining formal semantics to aid
us in reasoning about CLDs.
%
We have considered two such approaches: one based on qualitative
probabilistic networks and the other on difference equations.
%
We describe and compare these two approaches in the following
sections.

\subsubsection{Notation}
We denote a positive causal relationship between $A$ and $B$ by
$A\xrightarrow{+} B$ and a negative one by $A \xrightarrow{-} B$.
%
Then $A \xrightarrow{+} B$ informally means that an increase in $A$
causes an increase in $B$ (and a decrease in $A$ causes a decrease in
$B$).
%
On the other hand, $A\xrightarrow{-} B$ means that an increase in $A$
causes a decrease in $B$ (and conversely a decrease in $A$ causes an
increase in $B$).
%
We denote the sign of the edge from $A$ to $B$ by $s_{AB}$, so
$s_{AB}= +$ if $A\xrightarrow{+} B$ and $s_{AB}=-$ if
$A\xrightarrow{-} B$.

A vertex $A$ also has a sign $s_A$ that denotes the total influence on
$A$, so $s_A=+$ if there is an increase in $A$, $s_A=-$ if there is a
decrease, $s_A=0$ if there is no change and $s_A=?$ if we cannot
determine the change in $A$.

\subsection{Qualitative Probabilistic Networks}
One approach to modelling and reasoning about CLDs is by using
qualitative probabilistic networks (QPNs).

A QPN \cite{Wellman} is defined as a directed acyclic graph $G=(V,E)$
where the vertices, $V$, correspond to variables and the edges, $E$ to
qualitative probabilistic influences.
%
These influences can be positive (+) or negative (-).
%
The signs (?), for ambiguous influence, and (0), for probabilistic
independence, can also be used to describe probabilistic
relationships.

The meaning of signs on edges is defined according to first order
stochastic dominance, as follows:

Let $F_B(\cdot|a_i, x)$ be the cumulative distribution function (CDF)
for $B$ given $A=a_i$.
%
Then $s_{AB}=+$ means that for all possible values $a_1,a_2$ of $A$
where $a_1\geq a_2$, we must have:
%
\[F_B(b_0|a_1, x)\leq F_B(b_0|a_2, x),\]
%
that is,
%
\[P(B \leq b_0| A = a_1, x)\leq P(B\leq b_0| A = a_2, x)\]
%
for all possible values $b_0$ of $B$ and any consistent context $x$.
%
The context $x$ ranges over all possible assignments to the variables
other than $A$ that influence $B$, that are consistent with both
$A=a_1$ and $A=a_2$.
%
The definition of $s_{AB}=-$ is the same but with $a_1\leq a_2$.

In simpler terms, $s_{AB} = +$ means that greater values of $A$ mean
greater values of $B$ are more likely, and $s_{AB}=-$ means that
greater values of $A$ mean smaller values of $B$ are more likely.

These influences are symmetric, that is, if the edge from $X$ to $Y$
is reversed we must have $s_{XY} = s_{YX}$.
%
Due to this symmetry it is possible to propagate an observed increase
or decrease of one variable around the graph and find if other
variables are likely to have increased or decreased.

This definition is broad enough to apply to many different systems and
to be applicable to various real world situations.

\subsubsection{Issues} \label{qpnIssues}

We found some issues with QPNs that lead us to explore other
approaches.

First of all, since QPNs were originally defined for acyclic graphs
and the theory on them relies on acyclicity, they may not be the best
fit to describe CLDs, in which cycles (feedback loops) are an
important feature. However, it is possible to implement algorithms for inference
on QPNs containing loops, as is discussed in \cite{vanKouwen}.

% Comment from TC:
%It's not a matter of difficulty to implement.
%The key is the conditions under which the results are valid. I would include a
%reference to Van Kouwen

Second, the formal semantics of inference on QPNs is difficult to
formalize since it relies heavily on not-so-simple probability theory.
%
Additionally, QPNs are defined solely based on qualitative
relationships and there is no obvious way to expand them to also
describe quantitative relationships unless we have information about
the probability distributions and conditional probabilities involved.
%
GMB sessions will not produce data on probability distributions and
estimating such probabilities in a reliable manner requires a sizeable
dataset (that may not be available for a given GCM) as well as
statistical expertise.

Lastly, since all inference in QPNs is probabilistic it leads to
results that may not be as meaningful or concrete as we would like,
such as ``there is a heightened probability that $x$ has increased'',
rather than ``$x$ has increased''.
%
For instance, a variable may decrease even though the cumulative
probabilistic influence on it is positive.

\subsection{Difference equation approach}

Inspired by a system of tanks with water flowing from one to another,
and in search of semantics that might also be extended to quantitative
reasoning, we came up with the following approach.

We consider the values of the graph's vertices to be functions of the
same variable, such as a time variable $t$.

If we have a graph with two vertices, $X$ and $Y$, and one edge from
$X$ to $Y$, then $s_{XY}=+$ implies that
%
\[\frac{\partial Y}{\partial t} = G(X(t)),\]
%
where $G$ is a monotonically increasing function (monotonically decreasing for negative
causality, $s_{XY}=-$).

If the vertex $Y$ has multiple parent vertices $X_1,\ldots,X_n$, then
$\frac{\partial Y}{\partial t}$ depends on all the parent vertices.
%
We can isolate the effect of a single parent vertex $X$ on $Y$ by
differentiation.

In general we can then describe the causal relationship from $X$ to
$Y$ as
%
\[\frac{\partial\left( \frac{\partial Y(t)}{\partial t} \right)}{\partial Y(t)} =
  g(X(t)),\]
%
where $g$ has a primitive function $G$ such that $G$ is monotonically
increasing if $s_{XY}=+$ and monotonically decreasing if $s_{XY}= -$.

This is somewhat more nuanced than $CLDs$ as they are described above,
where $s_{XY}=+$ implies that an increase in $X$ leads to an increase
in $Y$, and a decrease in $X$ to a decrease in $Y$.
%
Here we may have some threshold value $x_0$ for $X$, where
$G(x_0) = 0$, above which $X$ always causes an increase in $Y$, but an
increase in $X$ causes a faster rate of increase in $Y$ and a decrease
in $X$ causes a slowed rate of increase in $Y$, and vice versa.

Note that though $G(X)$ is monotonically increasing, it may not be
strictly increasing, so we could for instance have $G(X) = 0$ for all
$X < C$ for some threshold value $C$.

If the vertex $Y$ has parent vertices $X_1,\ldots,X_n$, then we have
\[\frac{\partial Y}{\partial t} = \sum_{i=1}G_i(X_i),\]
where $G_i$ is monotonically increasing if $s_{X_iY}=+$ but monotonically
decreasing if $s_{X_iY}=-$.

In a discrete time system we consider $\Delta(X_t) = X_t - X_{t-1}$
instead of $\frac{\partial X}{\partial t}$, and write
$\Delta(X_t) = G(Y_{t-1})$ instead of
$\frac{\partial X}{\partial t} = G(Y(t))$.
%
In simple cases we may only consider one time step with two values of
$t$: $t_{start}$ and $t_{end}$.

Here we explore how this approach relates to qualitative reasoning,
but it could be extended to quantitative reasoning by solving the
appropriate differential equations.

\subsubsection{Simple qualitative model}

We consider a qualitative discrete time system where all values of
vertex variables are either +, -, 0, or ? (where ? is an ambiguous
value assigned to a variable whose value cannot be deduced).
%
These values have the partial ordering $- < 0 < +$, but ? cannot be
compared to the other values.
%
In place of addition and multiplication we have the operations
$\oplus$ and $\otimes$, whose behaviour can be seen in the following
tables:
\begin{center}
\begin{tabular}{c|cccc}
$\oplus$ & + & - & 0 & ?\\
\hline
  +   & +  & ? & + & ?\\
  -   & ?  & - & - & ?\\
  0   & +  & - & 0 & ?\\
  ?   & ?  & ? & ? & ?\\
\end{tabular}
\quad
\begin{tabular}{c|cccc}
$\otimes$ & + & - & 0 & ?\\
\hline
  +   & +  & - & 0 & ?\\
  -   & -  & + & 0 & ?\\
  0   & 0  & 0 & 0 & 0\\
  ?   & ?  & ? & 0 & ?\\
\end{tabular}
\end{center}
The only strictly increasing function in this system is the identity
function $id(x) = x$, and the only strictly decreasing function is the
negation function $neg(x) =-\otimes x$.

For simplicity we consider the case where all initial values are set
to zero and $G_e(0)=0$, for all edges $e$.
%
We only consider edge functions $G_e$ where $G_e(?) = ?$, since we
shouldn't be able to make unambiguous deductions based on ambiguous
values.
%

This is convenient for qualitative reasoning since then we are only
concerned with increases and decreases rather than numerical values.
%
The value of variable $X$ at time $t$, which we denote by $X_t$, then
tells us whether there has been a net increase or decrease in $X$.

Consider a graph with three vertices, $Z$ and its two parents $X$ and
$Y$, $X\xrightarrow{s_{XZ}} Z$ and $Y\xrightarrow{s_{YZ}} Z$.
%
Then we have
%
\[\Delta(Z_t) = G_{XZ}(X_{t-1}) \oplus G_{YZ}(Y_{t-1}),\]
%
where $G_{XZ}$ and $G_{YZ}$ are monotonically increasing or decreasing
in accordance with $s_{XZ}$ and $s_{YZ}$.
%
If we only allow strictly increasing/decreasing functions we then have
%
\[\Delta(Z_t) = (s_{XZ}\otimes X_{t-1})\oplus (s_{YZ}\otimes Y_{t-1}).\]

Consider a graph with three vertices $A$, $B$ and $C$ and two edges,
$A\xrightarrow{s_{AB}} B$ and $B\xrightarrow{s_{BC}} C$.
%
Then we have
\begin{align*}
\Delta(C_t) &= G_{BC}(B_{t-1})\\
&= G_{BC}(B_{t-2} \oplus \Delta(B_{t-1}))\\
&= G_{BC}(B_{t-2} \oplus G_{AB}(A_{t-2}))\\
\end{align*}

If $G_{BC}$ is linear, as is the case when we restrict the available
functions to the strictly increasing/decreasing $id$ and $neg$, we
then have
%
\[\Delta(C_t) = G_{BC}(B_{t-2})\oplus G_{BC}\circ G_{AB}(A_{t-2}).\]

If we only allow strictly increasing/decreasing functions we then have
%
\[\Delta(C_t) = (s_{BC}\otimes B_{t-2})\oplus (s_{BC}\otimes
  s_{AB}\otimes A_{t-2}).\]

\subsection{Comparison of approaches}

We achieve the same results when inferring on CLDs no matter whether
we use the QPN approach or the difference equation approach to
describe the underlying semantics.
%
Which method is simpler to understand and reason about is a matter of
opinion, but we encountered some difficulties when working with the
QPN approach that are outlined in section \ref{qpnIssues}, which make
us sceptical of the extensibility of that approach to quantitative
analysis.

We believe the difference equation method is well suited to describing
stock-and-flow diagrams as the approach was originally inspired by
considering stock-and-flow systems, and we are working towards
extending from CLDs to the more general GCMs.


\bibliographystyle{abbrv}
\bibliography{d4.2}

\end{document}
