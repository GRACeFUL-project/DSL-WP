\section{Background}
\label{sec:background}

We provide some background information about several aspects we use in
later sections, such that they become more accessible.
%
Whenever possible we give links for further reading as well.

\paragraph{\acl{DSL}} \citet{fowler} defines a \ac{DSL} as follows:
%
\emph{a computer programming language of limited expressiveness
  focussed on a particular domain}.
%
A \ac{DSL} is targetted at a specific class of programming tasks.
%
By restricting scope to a particular domain, one can tailor the
language specifically for that domain.
%
There are two main approaches to implementing \acp{DSL}:
\begin{description}
\item[standalone] A language with their own specialised syntax and
  parser to translate programs written in the \ac{DSL} into the host
  language.
%
  An advantage is that the syntax can be tailor made for the target
  audience, and does not have to resemble the host language.
%
  However, creating such a \ac{DSL} is labour intensive and it cannot
  easily reuse features, such as variables and conditionals, from the
  host language.
\item[embedded] An embedded language tries to offer a convenient
  syntax and abstraction mechanisms, but is offered as a library
  written in the host language.
%
  This means that all the existing facilities, such as abstractions
  standard libraries, are directly available.
%
  A disadvantage of an embedded \ac{DSL} is that the users may be
  unfamiliar with the host language.
\end{description}
\citet{Gibbons2015} gives a good overview of implementing \acp{DSL}
using functional programming.

\paragraph{Constraint programming} Similar to functional and logic
programming, constraint programming is a declarative programming
paradigm, which means that the order of processing is not fixed.
%
A constraint program is formulated in terms of a number of
constraints.
%
Such constraints are different from the common primitives of
imperative programming languages in that they do not specify a step or
sequence of steps to execute, but rather the properties of a solution
to be found.
%
The program that identifies the solutions satisfying these constraints
is called a onstraint solver.
%
In general there might be none, one, or many solutions to a particular
contraint problem.

The constraint programming approach is to search for a state in which
a large number of constraints are satisfied at the same time.
%
A problem is typically expressed as a state containing a number of
unknown variables.
%
The constraint solver searches for values for all the variables.
%
For more information, see, e.g., \url{http://constraint.org}.

\paragraph{\acfp{GCM}}
In a \acl{GMB} session stakeholders perform a policy analysis that
results in the form of a \acl{GCM}.
%
This concept map contains important elements of a policy problem
definition: goals, criteria for assessing the achievement of those
goals, a description of the system involving factors and criteria, and
the competing alternatives.

The term \emph{factors} refers to characteristics of a system that can
take a value, either quantitative or qualitative, that can change over
time (source: the \emph{GRACeFUL Glossary of Terms}).
%
Similarly, \emph{external factors} correspond to \emph{inputs} to the
system.
%
As such, factors are assumed to be associated with measurable
values.
%
As in systems theory, a distinction is made between inputs that are
beyond the control of the system, and which are potentially uncertain
(or even unknown), and \emph{actions} or combinations of actions,
which represent the system theoretical \emph{controls}.

The interaction between the factors, and between the factors and the
criteria, represents a first description of the system.
%
This description is in the form of a stock-and-flow diagram (which is
a generalisation of \aclp{CLD}~\cite{burns}).
%
A stocks-and-flow diagram represents the structural understanding of a
system --- the causal structures that produces the observed behavior.
%
It reveals information about the rates of change of system elements
and the measures of the variables of the system.
%
A diagram consists of the folowing elements:
\begin{description}
\item[Stocks] A stock represent a part of a system whose value at any
  given instant in time depends on the systems past behavior.
\item[Flows] Flows represent the rate at which the stock is changing
  at any given instant, they either flow into a stock (causing it to
  increase) or flow out of a stock (causing it to decrease).
\item[Converters] Converters either represent parts at the boundary of
  the system (i.e. parts whose value is not determined by the
  behaviour of the system itself) or they represent parts of a system
  whose value can be derived from other parts of the system at any
  time through some computational procedure.
\item[Connectors] Much like in causal loop diagrams the connectors of
  a system show how the parts of a system influence each other.
%
  Stocks can only be influenced by flows (i.e. there can be no
  connector that connects into a stock), flows can be influenced by
  stocks, other flows, and by converters.
%
  Converters either are not influenced at all (i.e. they are at the
  systems boundary) or are influenced by stocks, flows and other
  converters.
\item[Source/Sink] Sources and sinks are stocks that lie outside of
  the models boundary --- they are used to show that a stock is
  flowing from a source or into a sink that lies outside of the models
  boundary.
\end{description}
The elements of a policy problem are described in terms of
stock-and-flow diagram elements, for example \emph{factors} are
represented by \emph{stocks}.

TODO: explain this in much more detail.

(PaJa: no need to have ``much more detail'' for this deliverable. But
we need to connect the sections a bit better. )

%At this stage, if we can attribute qualitative values to the various nodes, the
%causal loop diagram describes a system that can make the object of a qualitative
%simulation. The criteria play the roles of constraints: the qualitative
%trajectories that lead outside the criteria can be pruned by the constraint
%programming system.

%The causal loop diagram is enhanced to a GRACeFUL concept map by the
%addition of extra nodes and non-causal links. The extra nodes represent
%actions, goals, alternatives, and stakeholders.

%Like all previous elements, the actions are also obtained from the
%stakeholders. As explained in the previous section, actions are the
%components of alternatives, among which will be found the policy
%solutions we are seeking. Alternatives are, according to the GRACeFUL
%Glossary of Terms, ``action or combination of actions that influence the
%values of criteria''. They correspond to Walker's ``alternative
%policies'', where a \emph{policy} is defined ``loosely'' as ``a set of
%actions taken to solve a problem''. As in the case of goal definition,
%the GRACeFUL system can suggest such ``atomic actions'' that are useful
%to the goals in the context of the problem at hand.

%The role of actions in the GRACeFUL concept maps is to give initial
%values to the factors and criteria they influence. They do not link via
%causal loops to these factors, since actions are not of the requisite
%type (they do not represent values that can increase or decrease).
%However, they will usually have non-causal links, representing
%constraining relationships to the factors.

%In the other cases of extra nodes (goals, alternatives, and
%stakeholders), the links represent primarily ``belongs to''
%relationships: there are links from stakeholders to goals, indicating
%which stakeholders have declared which goals; there are links from goals
%to the criteria by which they are assessed; finally, there are links
%between alternatives and the factors and criteria that they (i.e., their
%constituent atomic actions) influence.
