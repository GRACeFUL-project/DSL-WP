%% -- Installation instructions -----------------------------------------------
%% ----------------------------------------------------------------------------

The purpose of this section is to outline the process of installing
the GenericLibrary software and its require dependencies.
%
Section~\ref{install-overview} provides an overview of the software
dependencies required for development using GenericLibrary.
%
Section~\ref{install-docker} provides installation instructions for a
platform-independent package utilising the
\href{https://www.docker.com/}{Docker}
platform.
%
Section~\ref{install-minizinc} provides instructions for installing
and running the GenericLibrary software on UNIX-like platforms, for
users with an existing Haskell toolchain.

%% -- Software dependencies ---------------------------------------------------

\subsection{Software dependencies}
\label{install-overview}

The GenericLibrary language is implemented in
\href{https://www.haskell.org/}{Haskell} and uses the solver tools
from the \href{http://www.minizinc.org/}{MiniZinc} software
distribution.
%
Specifically, development and execution of GenericLibrary programs
requires the following software dependencies to be met:

\begin{itemize}
\item The MiniZinc and Gecode solver software.
%
  These can be found in the MiniZinc software bundle.
%
  Section~\ref{install-minizinc} outlines the process of installing
  these tools on UNIX-like systems.
\item A complete Haskell toolchain able to download packages from
  \href{https://hackage.haskell.org/}{Hackage}.
%
  Two alternative tools (Stack and Cabal) are suitable for this
  purpose, and both are provided by
  \href{https://www.haskell.org/platform/}{the Haskell Platform}.
\end{itemize}

Alternatively, for those who only wish to run the GenericLibrary
examples, we provide an alternative solution in the following section.

%% -- Using Docker ------------------------------------------------------------

\subsection{Installation using Docker}
\label{install-docker}

Download and install the Docker app from \url{https://www.docker.com/products/docker}.
%
Open a terminal (or the \emph{command prompt} under Windows) and
execute
\begin{verbatim}
  docker pull eugraceful/generic-library
  docker run --rm eugraceful/generic-library
\end{verbatim}

This will download and execute the example located at
\href{https://github.com/GRACeFUL-project/GenericLibrary/blob/master/examples/Examples.hs}{examples/Examples.hs}.

\subsection{Installation with an existing toolchain}
\label{install-minizinc}

This section provides instructions for installing the MiniZinc/Gecode
software dependencies on macOS and Linux systems.
%
Moreover, we provide instructions for building and running
GenericLibrary software using one of the existing toolchains mentioned
in Section~\ref{install-overview}.

%% -- Install MiniZinc: macOS -------------------------------------------------

\subsubsection{Installating MiniZinc/Gecode: macOS}

\begin{enumerate}
\item Download the complete MiniZinc
  distribution\footnote{\url{https://github.com/MiniZinc/MiniZincIDE/releases/download/2.1.2/MiniZincIDE-2.1.2-bundled.dmg}},
  and follow its installation instructions.
\item The MiniZinc/Gecode binaries will need to be added to your
  environment \verb+PATH+. From a terminal, run
%
    \begin{verbatim}
      export MZ=/Applications/MiniZincIDE.app/Contents/Resources
      echo export PATH=\"$MZ:\$PATH\" >> ~/.bashrc
      source ~/.bashrc
    \end{verbatim}
\end{enumerate}

%% -- Install MiniZinc: Linux -------------------------------------------------

\subsubsection{Installating MiniZinc/Gecode: Linux}

\begin{enumerate}
\item Download and install the complete MiniZinc
  distribution\footnote{\url{https://github.com/MiniZinc/MiniZincIDE/releases/download/2.1.0/MiniZincIDE-2.1.0-bundle-linux-x86_64.tgz}}.
%
  Place the contents of this file in a suitable directory, i.e.\
  \verb+~/MY_MZ_DIR/+.
\item Copy the solver tools from the extracted archive to a location
  on your path using
%
\begin{verbatim}
  cd ~/MY_MZ_DIR/
  cp solns2out mzn2fzn fzn-gecode mzn-gecode minizinc \
     ~/.local/bin
\end{verbatim}

\item Set the \verb+MZN_STDLIB_DIR+ environment variable:
%
\begin{verbatim}
  echo export MZN_STDLIB_DIR=~/MY_MZ_DIR/share/minizinc \
      >> ~/.bashrc
  source ~/.bashrc
\end{verbatim}
\end{enumerate}

%% -- Running a constraint program --------------------------------------------

\subsubsection{Running a constraint program}

With the MiniZinc/Gecode software dependencies met, we are now able to
build and execute GenericLibrary programs.
%
First, clone the GenericLibrary source repository.
%
From a terminal, run

\begin{verbatim}
  git clone https://github.com/GRACeFUL-project/GenericLibrary/
\end{verbatim}

\begin{itemize}
\item \textbf{Using cabal sandboxes}.
%
  Create a new sandbox and install the required dependencies with
%
\begin{verbatim}
  cabal sandbox init
  cabal install --dependencies-only
\end{verbatim}

  Build and execute the examples with

\begin{verbatim}
  cabal build
  cabal run examples
\end{verbatim}

\item \textbf{Using stack}.
%
  Stack will automatically take care of dependencies and
  sandboxing.
%
  Build and execute the examples with

\begin{verbatim}
  stack build
  stack exec examples
\end{verbatim}
\end{itemize}

This will build and execute the examples located in the
\href{https://github.com/GRACeFUL-project/GenericLibrary/blob/master/examples/Examples.hs}{examples/Examples.hs}
source file of the repository.
