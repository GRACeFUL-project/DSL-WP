%% -- Installation instructions -----------------------------------------------
%% ----------------------------------------------------------------------------

The GenericLibrary language is implemented in the 
\href{https://www.haskell.org/}{Haskell} programming language and utilises the 
solver tools from the \href{http://www.minizinc.org/}{MiniZinc} software
distribution. We provide a platform-independent package utilising 
\href{https://www.docker.com/}{Docker}. We also provide instructions for 
installing and running the GenericLibrary software on the macOS and Linux 
platforms for users with an existing Haskell toolchain.

%% -- Using Docker ------------------------------------------------------------

\paragraph{Installation using Docker.} Download and install the Docker app from
\url{https://www.docker.com/products/overview}. Open a terminal (or the 
\emph{command prompt} under Windows) and execute 

    \begin{verbatim}
      docker pull eugraceful/generic-library
      docker run --rm eugraceful/generic-library
    \end{verbatim}

This will download and execute the example located at
\url{https://github.com/GRACeFUL-project/GenericLibrary/blob/master/examples/%
Examples.hs}.

%% -- Install MiniZinc: macOS -------------------------------------------------

\paragraph{Installating MiniZinc/Gecode: macOS.} 

\begin{enumerate}
  \item Download the complete MiniZinc distribution from \url{https://%
    github.com/MiniZinc/MiniZincIDE/releases/download/2.1.2/MiniZincID%
    E-2.1.2-bundled.dmg}, and follow the installation instructions.
  \item The MiniZinc/Gecode binaries will need to be added to your environment 
    \verb+PATH+. From a terminal, run

    \begin{verbatim}
      export MZ=/Applications/MiniZincIDE.app/Contents/Resources
      echo export PATH=\"$MZ:\$PATH\" >> ~/.bashrc
      source ~/.bashrc
    \end{verbatim}
\end{enumerate}

%% -- Install MiniZinc: Linux -------------------------------------------------

\paragraph{Installating MiniZinc/Gecode: Linux.} 

\begin{enumerate}
  \item Download and install the complete MiniZinc distribution from
    \url{https://github.com/MiniZinc/MiniZincIDE/releases/download/2.1.0/%
    MiniZincIDE-2.1.0-bundle-linux-x86_64.tgz}. Extract this file in a suitable
    directory.
  \item Copy the solver tools from the extracted archive to somewhere on your
    path using

    \begin{verbatim}
      cd ~/YOUR_MINIZINC_DIR/
      cp solns2out mzn2fzn fzn-gecode mzn-gecode minizinc \
         ~/.local/bin
    \end{verbatim}

  \item Set the \verb+MZN_STDLIB_DIR+ environment variable:

    \begin{verbatim}
      export MZN_STDLIB_DIR=~/YOUR_MINIZINC_DIR/share/minizinc
    \end{verbatim}
\end{enumerate}

%% -- Running a constraint program --------------------------------------------

\paragraph{Running a constraint program} This section assumes that the user has
an existing Haskell toolchain consisting of the GHC 8 and either
\verb+cabal-install+ or \verb+stack+. First, clone the Generic Library source 
repository. From a terminal, run

\begin{verbatim}
  git clone https://github.com/GRACeFUL-project/GenericLibrary/
\end{verbatim}

\begin{itemize}
  \item \textbf{Using cabal sandboxes}. Create a new sandbox and install the
    required dependencies with

    \begin{verbatim}
      cabal sandbox init
      cabal install --dependencies-only
    \end{verbatim}

    Build and execute the examples with

    \begin{verbatim}
      cabal build
      cabal run examples
    \end{verbatim}

  \item \textbf{Using stack}. Stack will automatically take care of dependencies
    and sandboxing. Build and execute the examples with

    \begin{verbatim}
      stack build
      stack exec examples
    \end{verbatim}
\end{itemize}

This will build and execute the examples located in \href{https://github.com/%
GRACeFUL-project/GenericLibrary/blob/master/examples/%
Examples.hs}{src/examples/Examples.hs} source file of the repository.
