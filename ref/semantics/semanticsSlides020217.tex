\documentclass{beamer}
\usepackage[utf8]{inputenc}
\usepackage{amsmath,amssymb,graphicx,color,enumerate,hyperref}

\title{Formal semantics of GRACe}
\author{Sólrún Halla Einarsdóttir}
\institute{Chalmers University of Technology}
\date{2/2/2017}

\begin{document}
\begin{frame}
\titlepage
\end{frame}
\begin{frame}
\frametitle{Motivation/Background}
\begin{itemize}
  \item We would like to define formal semantics for our DSL
  \begin{itemize}
    \item Formal reasoning
    \item Robustness
    \item Gain further insights
  \end{itemize}

  \item We started by considering Causal Loop Diagrams, a specialisation of
the GCMs our DSL describes.

\item Work to extend and generalize these semantics is ongoing.

\end{itemize}
\end{frame}

\begin{frame}
\frametitle{Causal Loop Diagrams (CLDs)}
\begin{itemize}
\item A Causal Loop Diagram (CLD) is a directed graph used to display causal
relationships between variables.
%
\item Vertices represent variables, edges represent
qualitative causal relationships, which can be positive or negative.

\item CLDs have been modelled within GRACe.
%:see code file %\verb|QualitativeExample.hs|.
\end{itemize}
\end{frame}
\begin{frame}
\frametitle{Notation}
\begin{itemize}
\item We denote a positive causal relationship between $A$ and $B$ by
$A\xrightarrow{+} B$ and a negative one by $A \xrightarrow{-} B$.
%
\item We denote the sign of the edge from $A$ to $B$ by $s_{AB}$, so
$s_{AB}= +$ if $A\xrightarrow{+} B$ and $s_{AB}=-$ if
$A\xrightarrow{-} B$.

\item A vertex $A$ has a sign $s_A$ that denotes the total influence on
$A$, so $s_A=+$ if there is an increase in $A$, $s_A=-$ if there is a
decrease, $s_A=0$ if there is no change and $s_A=?$ if we cannot
determine the change.
\end{itemize}
\end{frame}

\begin{frame}
\frametitle{Qualitative Probabilistic Networks (QPNs)}
\begin{itemize}

\item A graph where vertices correspond to variables and edges to qualitative probabilistic influences.

\item $s_{AB} = +$ means that greater values of $A$ mean
greater values of $B$ are more likely, and $s_{AB}=-$ means that
greater values of $A$ mean smaller values of $B$ are more likely.

\item Possible to propagate an observed increase
or decrease of one variable around the graph and find if other
variables are likely to have increased or decreased.

\item Broad enough to apply to many different systems and
to be applicable to various real world situations.
\end{itemize}
\end{frame}

\begin{frame}
\frametitle{QPN Issues}
\begin{itemize}
\item QPNs were originally defined for acyclic graphs, theory relies on
  acyclicity, perhaps not the best
fit to describe CLDs, in which cycles (feedback loops) are an important feature.

\item Semantics of inference on QPNs is difficult to
formalize since it relies heavily on not-so-simple probability theory.

\item Can't be expanded to describe quantitative relationships unless we have information about
the probability distributions and conditional probabilities involved.

\item All inference in QPNs is probabilistic, may lead to
results that are not as meaningful or concrete as we would like,
\end{itemize}
\end{frame}

\begin{frame}
\frametitle{Difference equations}
\begin{itemize}
\item Inspired by a system of tanks with water flowing from one to another.

\item We consider the values of the graph's vertices to be functions of the
same variable, such as a time variable $t$.

\item We can describe the causal relationship from $X$ to $Y$ with
\[\frac{\partial Y}{\partial t} = G(X(t)),\]
where $G$ is monotonically increasing if $s_{XY}=+$ and monotonically decreasing if $s_{XY}= -$.
\item If the vertex $Y$ has parent vertices $X_1,\ldots,X_n$, we have
\[\frac{\partial Y}{\partial t} = \sum_{i=1}G_i(X_i(t)).\]

\end{itemize}

\end{frame}
\begin{frame}
\frametitle{Difference equations}
\begin{itemize}
\item More nuanced than classical CLDs, can have threshold values, rates of
  increase/decrease rather tahn just increase/decrease.

\item Monotonically increasing mustn't necessarily be strictly increasing.

%\item In a discrete time system we consider $\Delta(X_t) = X_t - X_{t-1}$
%instead of $\frac{\partial X}{\partial t}$, and write
%$\Delta(X_t) = G(Y_{t-1})$ instead of
%$\frac{\partial X}{\partial t} = G(Y(t))$.

\item Extend to quantitative reasoning by solving the
appropriate differential equations.

\end{itemize}
\end{frame}

\begin{frame}
\frametitle{Simplified qualitative difference equations}
We consider a qualitative discrete time system where all values of
vertex variables are either +, -, 0, or ?.
\begin{itemize}
\item Values have partial ordering $- < 0 < +$, but ? cannot be
compared to the other values.

\item The only strictly increasing function in this system is the identity
function, the only strictly decreasing function is the
negation function.

\item For simplicity we consider the case where all initial values are set
to zero.

\item The value of variable $X$ at time $t$, which we denote by $X_t$, then
tells us whether there has been a net increase or decrease in $X$.
\end{itemize}

\end{frame}
\begin{frame}
\frametitle{Comparison of approaches}
\begin{itemize}
\item Same results when inferring on CLDs no matter which approach is used to
describe the underlying semantics, which method is simpler to understand and reason about is a matter of
opinion.
\item Difficulties when working with the
QPN approach described earlier, we are skeptical about its extensibility.

\item Difference equation method was inspired by stock-and-flow-like systems,
  should extend well.
\end{itemize}
\end{frame}

\end{document}
\begin{frame}
\frametitle{}
\end{frame}